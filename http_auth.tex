\section{Аутентификация в веб-сервисах}
\selectlanguage{russian}

Протокол HTTP\index{протокол!HTTP} (вместе с HTTPS\index{протокол!HTTPS}) является, в настоящий момент, наиболее популярным протоколом, использующимся в сети Интернет для доступа к сервисам, таким как социальные сети или веб-клиенты электронной почты. Данный протокол является протоколом запрос-ответ\index{протокол!запрос-ответ}, причём для каждого запроса открывается новое соединение с сервером\footnote{Для версии протокола HTTP/1.0 существует неофициальное~\cite[p.~17]{Totty:2002} расширение в виде заголовка \texttt{Connection: Keep-Alive}, который позволяет использовать одно соединение для нескольких запросов. Версия протокола HTTP/1.1 по умолчанию~\cite[6.3.~Persistence]{rfc7230} устанавливает поддержку выполнения нескольких запросов в рамках одного соединения. Однако все запросы всё равно выполняются независимо друг от друга.}. То есть протокол HTTP не является сессионным протоколом\index{протокол!сессионный}. В связи с этим задачу аутентификации на веб-сервисах можно различить на задачи первичной и вторичной аутентификации. \textit{Первичной аутентификацией}\index{аутентификация!первичная} будем называть механизм обычной аутентификации пользователя в рамках некоторого HTTP-запроса, а \textit{вторичной}\index{аутентификация!вторичная} (или \textit{повторной}\index{аутентификация!повторная}) -- некоторый механизм подтверждения в рамках последующих HTTP-запросов, что пользователь уже был \textit{ранее} аутентифицирован веб-сервисом в рамках первичной аутентификации.

Аутентификация в веб-сервисах также бывает \textit{односторонней}\index{аутентификация!односторонняя} (как со стороны клиента, так и со стороны сервиса) и \textit{взаимной}\index{аутентификация!взаимная}. Под аутентификацией веб-сервиса обычно понимается возможность сервиса доказать клиенту, что он является именно тем веб-сервисом, к которому хочет получить доступ пользователь, а не его мошеннической подменой, созданной злоумышленниками. Для аутентификации веб-сервисов используется механизм сертификатов открытых ключей\index{сертификат открытого ключа} протокола HTTPS\index{протокол!HTTPS} с использованием инфраструктуры открытых ключей\index{инфраструктура открытых ключей} (см. раздел~\ref{chapter-public-key-infrastructure}).

При использовании протокола HTTPS\index{протокол!HTTPS} и наличии соответствующей поддержки со стороны веб-сервиса клиент также имеет возможность аутентифицировать себя с помощью своего сертификата открытого ключа\index{сертификат открытого ключа}. Данный механизм редко используется в публичных веб-сервисах, так как требует от клиента иметь на устройстве, с которого осуществляется доступ, файл сертификата открытого ключа.

\subsection{Первичная аутентификация по паролю}

Стандартная первичная аутентификация в современных веб-сервисах происходит обычной передачей логина и пароля в открытом виде по сети. Если SSL-соединение не используется, логин и пароль могут быть перехвачены. Даже при использовании SSL-соединения веб-приложение имеет доступ к паролю в открытом виде.

Более защищенным, но мало используемым способом аутентификации является вычисление хэша от пароля $m$, соли $s$ и псевдослучайных одноразовых меток $n_1, n_2$ с помощью JavaScript в браузере и отсылка по сети только результата вычисления хэша.
\[ \begin{array}{ll}
    \text{Браузер} ~\rightarrow~ \text{Сервис:} & \text{HTTP GET-запрос} \\
    \text{Браузер} ~\leftarrow~ \text{Сервис:}  & s ~\|~ n_1 \\
    \text{Браузер} ~\rightarrow~ \text{Сервис:} & n_2 ~\|~ h( h(s ~\|~ m) ~\|~ n_1 ~\|~ n_2) \\
\end{array} \]

Если веб-приложение хранит хэш от пароля и соли $h(s ~\|~ m)$, то пароль не может быть перехвачен ни по сети, ни веб-приложением.

В массовых интернет-сервисах пароли часто хранятся в открытом виде на сервере, что не является хорошей практикой для обеспечения защиты персональных данных пользователей.

\input{openid}

\input{cookies_auth}
