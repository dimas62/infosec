\section{Длины ключей}
\selectlanguage{russian}

В табл. \ref{tab:recommended-key-lengths} приведены битовые длины ключей для криптосистем.
%Традиционные рекомендации основаны на аппроксимации существующих алгоритмов для взлома на 10-30 лет вперед.

\begin{table}[!ht]
    \centering
    \caption{Минимальные длины ключей в битах по стандартам России и США\label{tab:recommended-key-lengths}}
    \resizebox{\textwidth}{!}{ \begin{tabular}{|l|c|c|c|c|}
        \hline
        & \multirow{2}{*}{\parbox{1.5cm}{Блоковые шифры, $K$}} & \multicolumn{3}{|c|}{Схема ЭП} \\
        \cline{3-5}
        & & \parbox{1.5cm}{RSA\index{криптосистема!RSA}, $n$} & \parbox{2.3cm}{Эллипт. кривые, порядок точки} & \parbox{3.5cm}{Эль-Гамаль\index{криптосистема!Эль-Гамаля} $\mod p$: модуль / порядок (под)группы} \\
        \hline \hline
        \multicolumn{5}{|c|}{Взломано} \\
        \hline
        Биты & 56 & 663 & 109 & 503  \\
        Конкурс & \textsc{DesChal} & RSA-200 & ECC2K-108 &  \\
        Год & 1997 & 2005 & 2000 &  \\
        \hline \hline
        \multicolumn{5}{|c|}{Стандарт России} \\
        \hline
        Биты & 256 &  & 255 & \\
        ГОСТ & 28147—89 & --- & 34.10-2001 & --- \\
        Год & 1989 & & 2001 & \\
%       \hline
%       \multicolumn{2}{|l|}{\parbox{4cm}{Россия: нелицензируемая деятельность}} & \multicolumn{4}{c|}{40} \\
        \hline \hline
        \multicolumn{5}{|c|}{Стандарт США} \\
        \hline
        Биты & 128-256 & 1024-3072 & 151-480 & 1024-3072/160-256 \\
        FIPS № & 197 & draft 186-3 & draft 186-3 & draft 186-3 \\
        Год & 2001 & 2006 & 2006 & 2006 \\
%       \hline
%       \multicolumn{2}{|l|}{\parbox{4cm}{США: экспортные ограничения до 2001 г.}} & 56 & 512 & 112 & 512/112 \\
%       \hline \hline
%       \multicolumn{2}{|l|}{Традиционные} & 80 & 1024 & 160 & 1024/160 \\
%       \cline{3-6}
%       \multicolumn{2}{|l|}{рекомендации} & 112 & 2048 & 224 & 2048/224 \\
%       \hline
%       \multicolumn{2}{|l|}{\parbox{4cm}{Рекомендация Lenstra, Verheul для 2010 г.}} & 78 & 1369 & 146-160 & 1369/138 \\
        \hline
    \end{tabular} }
\end{table}
%}\end{center}


\subsection*{Скорость вычисления ЭП}

Сравним производительность схем ЭП, чтобы продемонстрировать преимущества ЭП вида Эль-Гамаля\index{криптосистема!Эль-Гамаля} перед RSA\index{криптосистема!RSA} для больших ключей. В приложении показано, что в модульной арифметике по модулю числа $n$ с битовой длиной $k \simeq \log_2 n$ операции имеют битовую сложность:
\[ \begin{array}{lcl}
    a^b \mod n & - & O(k^3), \\
    ab \mod n, ~ a^{-1} \mod n & - & O(k^2), \\
    a+b \mod n & - & O(k). \\
\end{array} \]

Так как все описанные схемы ЭП используют возведение в степень по модулю, то битовая сложность -- $O(k^3)$. Оценки количества целочисленных $t$-разрядных умножений при вычислении ЭП имеют вид:
\begin{enumerate}
    \item RSA\index{криптосистема!RSA}:
        \[ (2 \log_2 n) \cdot \left( \frac{\log_2 n}{t} \right)^2; \]
    \item DSA\index{DSA} (Digital Signature Algorithm, стандарт США~\cite{FIPS-PUB-186-4}), вычисляемая по принципу Эль-Гамаля\index{криптосистема!Эль-Гамаля} по модулю $p$ и с порядком циклической подгруппы $q$:
        \[ (2 \log_2 q) \cdot \left( \frac{\log_2 p}{t} \right)^2; \]
    \item ГОСТ Р 34.10-2001\index{ГОСТ Р 34.10-2001} (стандарт России~\cite{GOST-2001}) и ECDSA\index{ECDSA} (Elliptic Curve Digital Signature Algorithm, стандарт США~\cite{FIPS-PUB-186-4}), вычисляемые по принципу Эль-Гамаля\index{криптосистема!Эль-Гамаля} в группе точек эллиптической кривой по модулю $p$:
        \[ (2 \log_2 p) \cdot 4 \cdot \left( \frac{\log_2 p}{t} \right)^2; \]
\end{enumerate}

В табл. \ref{tab:signature-rate} приведены оценки скорости вычисления ЭП (оценки числа умножений 64-битовых слов).

\begin{table}[!ht]
    \centering
    \caption{Оценочное число 64-битовых умножений для вычисления ЭП\label{tab:signature-rate}}
    \begin{tabular}{|c|l|c|}
        \hline
        ЭП & Оценочное число 64-битовых умножений \\
        \hline \hline
        RSA\index{криптосистема!RSA} 1024 & $(2 \cdot 1024) \cdot \left( \frac{1024}{64} \right)^2 \approx$ 500 000 \\
        RSA\index{криптосистема!RSA} 2048 & 4 000 000 \\
        RSA\index{криптосистема!RSA} 3072 & 14 000 000 \\
        RSA\index{криптосистема!RSA} 4096 & 34 000 000 \\
        \hline \hline
        DSA 1024/160 & $(2 \cdot 160) \cdot \left( \frac{1024}{64} \right)^2 \approx$ 82 000 \\
        DSA 3072/256 & 1 200 000 \\
        \hline \hline
        ECDSA 160 & $(2 \cdot 160) \cdot 4 \cdot \left( \frac{160}{64} \right)^2 \approx$ 8 000 \\
        ECDSA 512 & 260 000 \\
        \hline \hline
        ГОСТ Р 34.10-2001 & $(2 \cdot 256) \cdot 4 \cdot \left( \frac{256}{64} \right)^2 \approx$ 33 000 \\
        \hline
    \end{tabular}
\end{table}
